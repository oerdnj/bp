\documentclass[a4paper, 12pt, notitlepage, oneside, numbers=noenddot]{report}
\pagestyle{plain}
\frenchspacing
\usepackage[utf8]{inputenc}
\usepackage{czech}
\usepackage{a4wide}
\usepackage{graphicx}
\usepackage[a4paper, top=25mm, bottom=25mm, left=35mm, right=15mm]{geometry}
\usepackage{amsmath}
\usepackage{natbib}
\usepackage{setspace}
\usepackage{array}
\usepackage[footnotesize,bf]{caption}
\usepackage{ctable}
\onehalfspacing
\clubpenalty=10000
\widowpenalty=10000

%\renewcommand{\chaptername}{}

\title{Strategie při volbě partnera: obsahová analýza online inzerátů}
\author{Ondřej Surý}
\date{2010-04-30}

\def\refname{Literatura}
\renewcommand{\bibnumfmt}[1]{\textit{#1}}

\begin{document}

\pagestyle{empty}
\begin{titlepage}
  \flushleft{\includegraphics[scale=1.0]{logo.jpg}}

  \begin{center}
    \bf{Masarykova univerzita}\\
    \bf{Fakulta sociálních studií}\\
    \bf{Katedra psychologie}\\

    \vspace{2cm}

    Bakalářská práce\\
    obor psychologie

    \vspace{2cm}

    \large{\bf{STRATEGIE PŘI VOLBĚ PARTNERA:\\ OBSAHOVÁ ANALÝZA ONLINE INZERÁTŮ}}\\

    \normalsize

    \vspace{2cm}

    \noindent Vypracoval: Ondřej Surý\\
    \vspace{2cm}

    \noindent Vedoucí bakalářské práce: prof. PhDr. Petr Macek, CSc. \\
  
    \vfill

    Praha\\
    2010
  \end{center}
\end{titlepage}

\pagenumbering{roman}
\pagestyle{empty}
~~~
\vfill
\emph{Prohlašuji, že jsem práci vypracoval samostatně a~že jsem všechny použité informační zdroje uvedl v~seznamu literatury.}
\clearpage

Rád bych zde poděkoval Kačence za trpělivost, kterou se mnou měla
v~průběhu psaní bakalářské práce, Feelovi za nakopnutí a~podporu,
kterou mi po dobu celého studia poskytoval.  Dále děkuji své sestře
Lucii za cenné připomínky, kterou teď doháním, a~Vlastíkovi za
bleskový rychlokurz TeXu.

\clearpage
\thispagestyle{empty}
\tableofcontents
\clearpage
\thispagestyle{empty}
\listoftables
\clearpage

\pagestyle{plain}
\pagenumbering{arabic}
\setcounter{page}{1}

\chapter{Úvod}

Tématem této práce je problematika výběru partnera v~prostředí
internetu.  Jako nosnou teorii jsem si vybral evoluční psychologii,
zmíněny jsou i~některé další psychologické teoretické přístupy.
V~prá\-ci jsou analyzovány seznamovací inzeráty uveřejněné
prostřednictvím internetu a~výsledky jsou podrobeny kvantitativní
analýze.  Analýzou seznamovacích inzerátů se zabývalo již více
zahraničních studií, nicméně práce, které tuto tematiku analyzují
v~prostředí internetu, se objevují teprve v~poslední době
\citep{HallEtAl2010, EllisonEtAl2006, GibbsEtAl2006}.

Toto téma pokládám důležité nejen kvůli tomu, že internet
a~seznamování přes něj se čím dál tím více stává součástí běžného
života \citep{EllisonEtAl2006}, ale také protože téma výběru vhodného
partnera patří do oblasti interpersonálních vztahů, které jsou
pokládány za základní lidskou potřebu \citep{Slamenik2008}.

V~teoretické části mapuji základní pojmy, které jsou pro seznamování
na internetu relevantní.  Nejprve vymezuji, co jsou blízké osobní
vztahy a~jejich podmnožina partnerské vztahy.  Dále se zabývám
vybranými psychologickými přístupy k~teoriím výběru partnera,
především Mursteinovou S--V--R teorií.  Pokračuji popisem pohlavního
výběru, teo\-rií rodičovských investic a~návazností těchto teorií na
evoluční psychologii, především dopady na~strategie volby partnera.
V~následující části se krátce věnuji internetovým seznamovacím
serverům.  Na závěr teoretické části vymezuji výzkum\-ný pro\-blém.

Ve výzkumné části shrnuji relevantní teorie, výzkumnou otázku a~na
jejich základě evolučně-psychologických preferencí volby dlouhodobého
partnera stanovuji hypotézy.  Následně definuji výzkumný vzorek,
metody sběru a~kódování dat.  Pokračuji výsledkovou částí, na kterou
navazuje část Diskuse, kde shrnuji výsledky výzkumu a~pokládám otázky
relevantní pro další výzkum.

V~závěru práce shrnuji nejdůležitější poznatky této práce.

\chapter[Teoretická část]{Teoretická část}
\section{Blízké osobní vztahy}
Blízké osobní vztahy (anglicky Close relationships) jsou důležitou
oblastí lidských vztahů a~jsou důležitou součástí lidské společnosti.
Vztahy mezi lidmi jsou téměř tak důležité pro přežití jako je vzduch
a~voda -- vztah rodiče a~dítěte, dvou milenců, či dvou přátel.  Blízké
osobní vztahy mají různou podobu, různou míru důvěrnosti i různou
formu.  Procházejí různorodými interpersonálními procesy, ohrožují je
krize a~jiné hrozby, a~můžeme se na ně dívat různými pohledy
\citep{HendrickHendrick2000}.

Blízké osobní vztahy patří do oblasti zkoumání sociální psychologie,
ale zasahují i~do ostatních oblastí psychologie, které jsou osobními
vztahy buď ovlivňovány nebo je ovlivňují.  Ve vývojové psychologii je
důležitý vztah mezi matkou, resp. primární pečující osobou, a~dítětem,
který následně ovlivňuje i~další vztahy v~dospělosti.  Blízké osobní
vztahy také zasahují do psychologie emocí.  Bowlby (dle
\citealp{GuerreroAndersen2000}) poukázal na to, že nejvíce emocí
zažíváme právě při navazování, udržování, obnovování, narušení nebo
ukončení blízkých osobních vztahů s~ostatními.
\citet{GuerreroAndersen2000} tuto myšlenku následně rozšiřují návrhem,
že interpersonální komunikace je primární spouštěč většiny emocí.

\citet{Kelley1986} definuje osobní vztahy (personal relationships)
v~první řadě jako ty, které jsou blízké, a~odvozením z~blízkosti jsou
také osobní.  Blízké vztahy pak definuje jako takové, ve kterých
jedinci ovlivňují sebe navzájem často, silně, různými způsoby
a~v~průběhu významně dlouhých časových intervalů.  Dle teorie vzájemné
závislosti jsou blízké osoby vysoce závislé v~průběhu významně
dlouhého časového období.

\citeauthor{Kelley1986} také zmiňuje, že blízkost vztahu neznamená, že
tento vztah bude bez problémů.  Vztah může mít vysokou míru závislosti
a~přesto mít problémy jako je konflikt zájmů, nerovného postavení nebo
problémů vyplývajících z~různých vzorů vzájemné závislosti.

\section{Partnerské vztahy}
Speciální podoblastí blízkých osobních vztahů jsou partnerské vztahy.
Dle Vágnerové \citeyearpar{Vagnerova2007} je partnerství
nejdůležitější variantou intimního vztahu, v~němž se člověk dokáže
vzdát části své samostatné identity.  Partnerství, tj. párová
identita, je za takových okolností víc než individuální identita a~víc
než pouhá koexistence dvou lidí.  Takový vztah má jinou kvalitu, než
měly předchozí milostné experimenty, a~proto se může stát
perspektivním základem dalšího párového i~osobního vývoje.

Období vytváření partnerských vztahů je především v~první polovině
dospělosti.  \citet{Erikson1999} do období mladší dospělosti klade
jeden z~vývojových úkolů -- dosažení intimity pomocí prožitku
intimního vztahu.  \citet{Rican2006} definuje intimitu jako hodnotný
milostný vztah mezi mužem a~ženou a~v~silném psychologickém smyslu ji
definuje v~bodech takto:

\begin{enumerate}
\item Tělesná a~duševní něha, která není brzděna ani ostychem, ani
  opatrností a~strachem ze závislosti nebo ze zesměšnění, ani
  rozumovými důvody nebo společenskými konvencemi.
\item Sebeotevření, přání dát se poznat ve svých silných i~slabých
  stránkách, nemít tajemství.
\item Důvěra bez výhrad.
\item Vzájemná úcta, respekt.
\item Pravdivé poznání partnera.  Intimita znamená schopnost milovat
  ho i~s~jeho chybami.
\item Pravdivé poznání vlastních citů k~partnerovi.
\item Společné dílo, záměry a~plány, sdílená budoucnost.
\item Hravost, tvořivost, smysl pro humor.
\item Objevování, úžas, obdiv.
\item Výlučnost.  Intimita mezi mužem a~ženou je možná jen ve dvou.
\end{enumerate}

Vymezení hranic mladší dospělosti je ještě o~něco komplikovanější.  Na
dolní hranici, která je především vymezená ukončením tělesného vývoje
a~dosažením psychosociální vyspělosti jedince, se většina autorů
shodne.  Nicméně vymezení horní hranice se u jednotlivých autorů
poměrně liší.  Jednotliví autoři vymezují hranice dospělosti takto:

\begin{itemize}
\item \citet{LangmeierKrejcirova2007} dělí dospělost na časnou (zhruba
  od 20 do 25--30 let), střední (asi do 45 let) a~pozdní (asi do 60--65
  let).
\item Švancara (dle Langmeiera \& Krejčířové
  \citeyearpar{LangmeierKrejcirova2007}) dělí dospělost na mladou (od
  20 do 30--32 let), střední (od 30--33 do 46--48 let) a~starší (do 60--65
  let).
\item \citet{Rican2006} dělí dospělost na ,,Zlatá léta dvacátá`` (od 20 do
  30 let), ,,Životní poledne`` (od 30 do 40 let), ,,Druhý dech`` (od
  40 do 50 let), ,,Jaké je to po padesátce`` (od 50 do 60 let).
\item \citet{Vagnerova2007} dělí dospělost na mladou (20 až 40 let),
  střední (40--50 let) a~starší (50 až 60 let).
\end{itemize}
Pro potřeby této práce se jeví jako vhodné používat dělení dle
Vágnerové, kde horní hra\-nice období mladé dospělosti přibližně
odpovídá konci plodného věku reálného \citep{CSU2006,CSU2003}
i~biologického (45--50 let) \citep{Rozsypal2003}.  Jedinci ve věku
střední a~starší dospělosti budou na potenciálního partnera
pravděpodobně klást již odlišné požadavky, protože vlastní potomky již
mají nebo potomky již mít nemohou.  Změnu v~požadavcích na
potenciálního partnera potvrzuje například
\citet{Gil-BurmanPelaezSanchez2002} na vzorku španělských mužů a~žen.
Muži pod čtyřicet let nabízí především fyzickou atraktivitu, nad
čtyřicet let již zdroje a~socioekonomický status.  Španělské ženy
starší čtyřiceti let hledají především socioekonomický status ~~a ty
pod~~ čtyřicet let fyzickou přitažlivost.

\subsection{Výběr partnera}
Partnerství a~výběr vhodného partnera patří k~důležitým životním
úkolům člověka.  K~po\-če\-tí potomků a~tedy k~zajištění pokračování
lidského rodu jsou zapotřebí dva plodní jedinci opačného pohlaví.
I~přesto, že moderní doba nabízí mnohé alternativní metody početí
(např. IVF), přirozeným a~nejčastějším způsobem je počít potomka
v~dlouhodobém partnerském vztahu \citep{CSU2006}.

Dle Výrosta \citeyearpar{Vyrost2008} je sice většina literatury o~osobních
vztazích v~dyádách založena na úplné svobodě výběru partnera, ale ve
skutečnosti země, jejich kultura a~tradice, umožňují tento postup, jsou
ve výrazné menšině. Další alternativy výběru partnera mohou být:

\begin{itemize}
\item Výběr partnera je výsadním právem rodiny (rodičů) bez možnosti
  páru ovlivnit výsledek
\item Rodina, resp. rodiče, uskutečňují výběr, ale jeden (častěji se
  vyskytující případ) nebo oba členové dyády mají právo toto
  rozhodnutí vetovat
\item Partnera si členové dyády vybírají sami, ale rodina, resp. rodiče,
  mají právo veta
\end{itemize}

\subsection{Hledání partnera}
Na faktory ovlivňující výběr a~volbu partnera je možné dívat se z~více
různých pohledů.  Faktory mohou být vědomé nebo nevědomé, vrozené nebo
naučené, psychologické nebo socio-kulturní.  Dle Plaňavy \citeyearpar{Planava1998}
je možné nahlížet na hledání partnera psychoanalyticky, sociologicky,
psychologicky nebo pohledem syntézy předchozích přístupů (S-V-R
model).  V~následující části jsou představeny některé
z~psychologických přístupů k~hledání partnera.

\paragraph{Psychoanalytický přístup}
Freud \citep{Planava1998} nahlíží na hledání partnera jako na
podvědomý proces, kdy si člověk může uvědomovat, co činí, ale netuší,
proč tak činní.  Freud rozlišoval dva přístupy výběru sexuálního
partnera: v~prvním člověk hledá osobu, která by mu nahradila matku,
takovou volbu nazývá závislostní nebo anakly\-tic\-kou, v~druhém případě
hledá osobu, kterou by mohl milovat jako sám sebe, takovou volbu
nazývá narcistní.  Richter, jeden z~Freudových žáků, tuto teorii
rozvinul a~rozlišil pět modelů nevědomých očekávání, které v~sobě
neseme z~dětství a~které se rozhodujícím způsobem podílejí na výběru
partnera. Těchto pět modelů je založeno na nedostatkových až
traumatizujících zkušenostech a~prožitcích.

\paragraph{Sociologicko-psychologický přístup}
Jedna z~hlavních myšlenek sociologických pří\-stu\-pů \citep{Planava1998}
k~výběru a~volbě partnera říká, že se vyhledávají a~preferují \hbox{osoby}
pocházející ze stejného či blízkého prostředí, kulturního,
vzdělanost\-ního, sociál\-ního, nábožen\-ského.  Společensky si blízcí lidé
se častěji vyskytují pohromadě, pravděpodobnost seznámení je tudíž
vyšší.

Burges a~Locke \citep{Planava1998} provedli analýzu publikovaných
výzkumů a~studií na téma seznamování a~došli k~názoru, že nejčastěji
působí tyto základní faktory a~determinanty:

\begin{enumerate}
\item blízkost a~spřízněnost, to jest společná příslušnost k~určité
  sociální skupině
\item tlak nesouhlasu, pokud si někdo chce vzít někoho mimo vlastní
  skupinu či vrstvu
\item představy o~ideálním partnerovi
\item psychologická podobnost potenciálního partnera s~jedním z~rodičů
\item tak zvaná homogamie, to jest tendence, kterou by
  psychoanalytikové nazvali narcistní, neboli vrána k~vráně sedá,
  stejný stejného hledá
\end{enumerate}
\paragraph[S{}--V{}--R Teorie]{S-V-R Teorie}
\citet{Murstein1970} rozvinul a~empiricky otestoval teorii
Stimulus -- Value -- Role. Tato teorie chápe vývoj vztahu jako postupné
kroky jdoucí za sebou.

První a~z~hlediska tohoto výzkumu nejdůležitější první fáze byly
pojmenována Stimulus (neboli Podnět).  V~této fázi ještě samostatní
jedinci na sebe navzájem působí podněty.  První jedinec upoutá
pozornost druhého jedince, a~pokud je zájem opětován, působí na sebe
navzájem.  Typicky jsou dostupné informace o~druhé osobě velmi
omezené a~největší roli hrají vizuální a~sluchové indície. Druhý může
být posouzen, že má ,,sexy hlas``, je ,,příliš mladý``, ,,tak akorát``
nebo ,,příliš starý``.  Velkou roli tedy hraje fyzická atraktivita.

Pokud je jedinec vnímán jako nepřitažlivý, může být jeho nižší fyzická
atraktivita vykompenzována řadou jiných vlastností.  Murstein uvádí
příklad drsného ale ošklivého hráče ragby, který může být přitažlivý
díky představě mužného přitažlivého hrdiny, navíc s~možností slušného
finančního zajištění.  Stejně tak může jako kompenzační faktor působit
společenské postavení, reputace, postoje nebo náboženství.

Pokud by ovšem naše volba partnera záležela pouze na vzájemné
přitažlivosti, tak bychom skončili s~populací, která by měla jen málo
provdaných párů, protože by několik nejpřitažlivějších jedinců na sebe
strhlo veškerou pozornost ostatních.  Murstein uvádí další dva
faktory, které ovlivňují naši volbu partnera: za prvé posouzení
vlastní at\-rak\-ti\-vi\-ty pro druhé a~za druhé konceptualizace výběru
partnera jako směnného trhu. Jedinec se snaží oslovit takového
partnera, který je přibližně na stejné úrovni atraktivity, což jedinci
poskytne nejlepší poměr nákladů a~odměn. Tento přístup podporují
i~další teorie atraktivity, které připisují podstatnou roli
kognitivním procesům, kam (dle \citealp{Slamenik2008}) patří
např. teorie rovnováhy (Heider), teorie sociální výměny (Thibauta \&
Kelley) nebo Teorie rovnosti (Walster, Walster \& Berscheid).

Fáze Value, tedy hodnotová, je v~pořadí druhá.  V~této fázi partneři
hledají shodu v~dů\-le\-ži\-tých hodnotách, jejichž vzájemnou
kompatibilitu porovnávají na základě verbální komunikace.  Důležitá je
osobní atraktivita, která zahrnuje osobnostní vlastnosti, postoje,
názory, přesvědčení, hodnoty, zájmy, potřeby, víru, způsob života
i~socioekonomický status.  Pokud partneři úspěšně projdou touto fází,
následuje fáze poslední -- Role.

Ve fázi porovnávání rolí partneři hodnotí reálné fungování ve vztahu,
jak se shodují v~sexualitě, jak dokáží reagovat na vzájemné rozlady,
jak jsou schopni se podporovat, atd.  Pokud partnerský vztah stále
poskytuje dostatečnou úroveň odměn v~porovnání s~náklady do vztahu,
partnerský vztah by měl nadále pokračovat a~vyústit v~manželství
\citep{Murstein1970}.

\section{Pohlavní výběr}
Darwinova evoluční teorie popisuje dva mechanismy selekce, na jejichž
základě funguje -- přirozený výběr a~pohlavní výběr.  Přirozený výběr
popisuje vnitrodruhovou selekci na základě prostředí.  Jedná se
o~mechanismy, které umožňují jedinci přežít, jako jsou shánění potravy
nebo vyhýbání se predátorům \citep{Darwin2005, Buss2007}.  Darwin si
uvědomil, že některé znaky, jako je například paví ocas, okázalá pírka
kardinálů červených nebo enormní jelení paroží, představují pro své
nositele z pohledu šance pro přežití závažnou nevýhodu.  Například paví
ocas extrémně ztěžuje létání a~tím také možnost utéct před predátory.
Darwin si také povšiml, že samci a~samičky některých druhů mají
rozdílnou velikost, váhu nebo tvar.  Paviání samičky jsou o polovinu
menší než samci.  Mezi lidmi jsou muži v~průměru o~12\% vyšší než
ženy.

Darwin předpokládal, že tyto znaky se mohly vyvinout proto, že
znamenají pro dané zvíře při hledání pohlavního partnera výhodu, která
by mohla kompenzovat nižší pravděpodobnost přežití.  Pohlavní výběr
obsahuje dva základní principy.  První princip, který se nazývá
\underline{vnitropohlavní soutěž}, popisuje soupeření jednoho pohlaví
(většinou samců) mezi sebou.  Vítěz souboje získává přednostní přístup
k~samičkám.  Nutné je poznamenat, že tyto souboje nemusí být nutně
přímý fyzický zápas.  Samci některých druhů mezi sebou soupeří
o~postavení mezi ostatními samci nenásilným způsobem.  Klíčové je, aby
dědičné kvality vedly k~reprodukčnímu úspěchu.  Tyto kvality mohou být
fyzická síla, schopnost získat území nebo vyjednávací schopnosti
v~hierarchickém žebříčku.  Druhý proces pohlavního výběru je
\underline{mezipohlavní soutěž}, která zahrnuje preference druhého
po\-hla\-ví pro výběr partnera, který oplývá konkrétními kvalitami
\citep{Buss2007}.

Pohlavní výběr vysvětluje i~barevný ocas, který je preferovaný pavími
slípkami, ale také rozdílnou velikost samců (např. u rypoušů sloních
je samec čtyřikrát až pětkrát větší než samice) nebo jelení paroží
(jeleni s~mohutnějším parožím mají výhodu ve vnitropohlavním souboji,
ale také v~mezipohlavním -- laně preferují samce s~velkým parožím)
\citep{Buss2007}.

Rysy, které usnadňují živočichům zajistit si partnera (nebo mnoho
partnerů), se v~po\-pu\-laci budou šířit.  Podle teorie pohlavního
výběru si jedinci, kteří jsou nositeli takových rysů, zajistí větší
reprodukční úspěch, než ti, kteří jejich nositeli nejsou.  A také
vyšší úspěšnost s~jakou jsou určité vlastnosti či rysy rozšiřovány
v~budoucích generacích \citep{BarrettDunbarLycett2007}.  Tuto
úspěšnost budeme v~následujícím textu označovat jako fitness.

Důležité je poznamenat, že pro zachování genetické linie
z~dlouhodobého evolučního hlediska není důležitý jen počet přímých
narozených potomků, ale také počet potomků, kteří dosáhnou
reprodukčního věku a~mají vlastní potomky. Pro zachování genetické
linie je důležitá kontinuita.

Na dvě otázky ovšem Darwinova teorie pohlavního výběru ve své době
neměla odpověď.  Již samotný Darwin pozoroval, že samice jsou často
vybíravější ve volbě partnera a~samci jsou často tím soupeřivým
pohlavím.

\citet{Trivers1972} na obě tyto otázky odpověděl teorií rodičovských
investic.  U~pohlavně se rozmnožujících druhů investují vždy samice
více než samci, protože vajíčka jsou větší a~nákladnější než spermie.
U~savců je situace ještě více zvýrazněná existencí vnitřního oplodnění
a~březostí, takže samice zpočátku doslova ,,nosí potomka``.  Za podmínek,
kdy samice investují do potomků výrazně více, soupeří samci mezi sebou
o~přístup k~samicím a~takové soupeření zvyšuje možnost pohlavního
výběru.  Díky rozdílu v~investicích do zplození nového potomka jde při
výběru partnera samcům především o~kvantitu partnerek a~samicím
o~kvalitu partnerů.

Pokud se budeme bavit o~lidech, tak rozdíl ve velikosti investic do
zplození potomka také platí.  Investice ze strany muže může být
v~minimalistické podobě prezentována pouze jednou čajovou lžičkou
ejakulátu a~muž se může stát otcem při každém pohlavním styku s~novou
ženou \citep{Buss2007}.  Investice ze strany ženy je celých 38 týdnů
těhotenství a~minimálně během této doby žena nemůže počít další dítě.
Kojení dítěte způsobuje produkci prolaktinu, který blokuje ovulaci,
a~tím snižuje pravděpodobnost dalšího početí \citep{Kovar2005}.

Roku 1948 provedl britský vědec A. J. Bateman experiment s~octomilkami
(\textit{Drosophila melanogaster}).  Nastavil podmínky tak, aby se
octomilky mohly navzájem dle libosti pářit.  Jednalo se o~pět samců
a~pět samiček, které měly chromozomální markery, které následně
Bateman hledal u~potomků.  Zjistil, že nejplodnější samičky nebyly
o~mnoho úspěšnější než samičky nejméně plodné, kdežto nejplodnější
samci byly nesrovnatelně úspěšnější než ti nejméně plodní (Bateman,
dle \citealp{Ridley2007,Trivers1972}).  Tato asymetrie pozorovaná
Batemanem se dále prohlubuje s~vývojem rodičovské péče, která dosahuje
svého vrcholu u~savců.  Savčí samice rodí poměrně velké mládě, dlouho
je živí a~chrání ve svém nitru; samec se může stát otcem během
několika sekund.  Jestliže samice pojme větší počet partnerů, nijak
tím nezvýší svou plodnost, zatímco samec ano.  Pravidlo zjištěné
u~octomilek platí i~pro savce včetně lidí \citet{Ridley2007,Buss2007}.

Na základě tohoto pravidla byl formulován Batemanův princip, který
poskytuje dů\-le\-ži\-tý výchozí bod pro jakoukoli představu o~volbě
partnera.  Bateman \citep{BarrettDunbarLycett2007} zdůraznil, že
ikdyž průměrný celoživotní reprodukční úspěch obou pohlaví musí být
shodný, rozptyl celoživotního reprodukčního úspěchu se může znatelně
lišit, zvláště když existuje sklon k~polygamii.  Když rozptyl
celoživotního reprodukčního úspěchu u~jednoho pohlaví převýší tento
rozptyl u~druhého, bude pro pohlaví s~menší variabilitou výhodné stát
se vybíravější.  Následkem toho si může dovolit vyvinout větší tlak na
druhé pohlaví, které není schopno si vytvořit efektivní
protistrategii.

Investice do potomka jsou asymetrické mezi pohlavími, ať už velikostí
gamet \citep{Dawkins1998}, délkou těhotenství, či výchovou potomka.
Na základě této rozdílnosti mají muži a~ženy odlišné strategie volby
partnera \citep{BussSchmitt1993}.  Vzhledem k~těmto ex\-trém\-ním
nákladům dlouhotrvajícího těhotenství a~kojení potřebným pro růst
našich velkých mozků, jsou výrazně silné dvě posilující podmínky
Batemanova principu -- (a)~náklady na roz\-mno\-žo\-vá\-ní jsou vyšší
pro pohlaví s~nižší variabilitou a~(b) členové variabilnějšího pohlaví
se výrazně liší v~hodnotě jako partneři.  Kombinace těchto vlivů
vytváří předpoklad, že ženy budou vybíravější než muži v~tom, s~kým se budou
rozmnožovat.  Druhý následek nákladnosti lidského rozmnožování je
tendence žen vybírat si muže podle jejich vlivu na úspěch, se kterým
může být vychován potomek.  Protože výchova nového potomka je náročná
na zdroje, je volba partnera ovlivňována nejen ,,kvalitou`` genů, ale
také schopnosti poskytnout tyto zdroje pro výchovu potomka do
reprodukčního věku.  Úspěch může být v~této souvislosti dosažen dvěma
způsoby -- na základě kvalitních genů, které je muž schopen
poskytnout, nebo na základě schopnosti muže přispět k~rodičovské péči.
Základní evoluční principy tudíž předpokládají, že žena si bude volit
muže buď podle znaků genetické kvality, nebo podle jeho ochoty či
schopnosti přispět k~výchově dětí \citep{BarrettDunbarLycett2007}.

\subsection{Teoretické přístupy k~pohlavnímu výběru}
Pohlavně vybírané znaky jsou formou komunikace a~představují dva
propojené problémy: za prvé: jak dám najevo svou hodnotu jako
potenciální partner, za druhé: jak si mezi dostupnými partnery
vyberu toho s~nejvyšší hodnotou.  Několik vybraných alternativních
(nikoli vylučujících se) přístupů se snaží odpovědět na tyto otázky
(volně dle \citealp{Barber1995} a~\citealp{Ridley2007}):

\paragraph{Fisherova utíkající (runaway) teorie pohlavního výběru.}
Darwin správně identifikoval, že některé znaky, které mohou být
preferovány pohlavním výběrem, zároveň mohou být pro samečky
nebezpečné, ale již neodpověděl na otázku, jak se tyto znaky mohly
vyvinout.  Fisher navrhl teorii, že samičky pářící se samečky, kteří
vlastní dědičné znaky obecně přitažlivé pro samičky, budou mít
,,sexy`` syny a~dcery preferující tyto znaky.  Tímto způsobem dojde
k~tomu, že konkrétní znaky se budou stávat více a~více extrémními.

\paragraph{Pohlavní výběr dobrých genů.}
Přehnané pohlavní znaky znevýhodňují nositele (velké parohy) a~ten
musí být ,,zdravý`` -- mít dobré geny zajišťující přežití.  Pouze ten
nejsilnější a~nezdravější jelen, který je schopný najít dostatek
potravy a~vyhnout se predátorům si může ,,dovolit`` nosit velké
paroží.  Moderní teorie tento přístup modifikují směrem k~prostředí --
všem jelenům mohou vyrůst velké parohy, ale povede se to pouze těm,
kteří získají dostatek zdrojů.  Velké paroží je pravdivou demonstrací
fenotypové kvality a~jejího genetického základu. Ovlivnění vlastnosti
(velké paroží) interakcí mezi genotypem a~prostředím je vyřešen
problém fixace sady genů (tj. znaky se nemusí fixovat, protože
prostředí je dostatečně variabilní).

\paragraph{Pohlavní signály jako znak dědičné odolnosti proti
  patogenům.}
Hamilton \& Zuk tvrdí, že pohlavní signály nejsou ani tak indikátorem
toho, že je sameček schopný obstarat zdroje, jako že je málo zamořený
patogeny (viry, bakterie, paraziti). Teorii podpírá zjištění, že druhy
s~vysokou zamořeností parazity mají hodně výrazné pohlavní znaky.
Tuto teorii potvrzuje například studie, kterou provedli
\citet{GangestadBuss1993}.  Zkoumali vztah mezi zamořením populace
patogeny a~hodnocením fyzické přitažlivosti.  V~populacích, které byly
hodně postiženy patogeny, byla fyzická přitažlivost mnohem
důležitějším faktorem volby partnera než v~populacích, které nebyly
tolik postiženy.

\paragraph{Pohlavní signály jako znak imunitní zdatnosti a~vývojové
  stability.}
Tento přístup je rozšířením předchozí teorie o~patogenech. Patogeny
mohou v~průběhu vývoje u~jedinců s~nefunkčním nebo defektním imunitním
systémem narušit symetrii těla. Testosteron má schopnost potlačovat
imunitu, a~proto sameček s~výraznými a~symetrickými pohlavními znaky
(tedy s~vysokou hladinou testosteronu, který je zodpovědný za mnoho
sekundárních pohlavních znaků) prokazuje, že má vysokou imunitní
zdatnost.  Vysoká kompetence imunitního systému je obecně lepší
u~heterozygotů, a~protože jsou heterozygoti spojeni s~fenotypovou
průměrností, tak vlastnosti, které jsou průměrné, budou
nejpřitažlivější.  Tento efekt prokázaly i~další výzkumy, kdy jako
nejpřitažlivější byl shledáván obličej, který byl vytvořen jako průměr
mnoha ženských obličejů \citep{Ridley2007}.

\section{Evoluční psychologie}
Evoluční psychologie (anglicky: Evolutionary psychology) má své kořeny
ve sloučení dvou intelektuálních proudů, které byly v~minulosti
propojeny jen velmi okrajově: evoluční biologie a~psychologie
\citep{DunbarBarrett2009}.  Evoluční psychologie je stále mladá
vědecká disciplína, která se nechce stavět po bok ostatním podoblastem
v~psychologii jako je například kognitivní, sociální, vývojová, atd.,
ale která chce poskytnout teoretický (meta)rámec pro všechny tyto
oblasti.  \citet{Handbook2005} píše, že lidskou mysl nelze
ve dlouhém sporu ,,culture vs. nature`` nadále vnímat jako mysl
nepopsanou (,,tabula rasa``).  Místo toho lidská mysl přichází na svět
vybavena širokou škálou specializovaných psychologických mechanismů
vytvářených po dlouhou dobu přírodním a~pohlavním vý\-běr\-em, aby mohla
řešit stovky opakujících se problémů, se kterými se naši předkové
setkávali.

\citet{DunbarBarrett2009} k~tomuto dodávají, že je nutné se také
zaměřit na porozumění, jakou roli hraje lidská kultura, která umožňuje
sdílení vědomostí.  Od zvířat lišíme především kapacitou této sdílené
vědomosti, rozlišováním společenských faktů, používáním těchto
mechanismů pro vytváření společenských vztahů.  Mameli (cit. dle
\citealp{DunbarBarrett2009}) také upozorňuje na důležitost kulturního
přenosu, který u\-mo\-žňu\-je mezigenerační sdílení vědomostí.  Naše
odlišnost od zvířat však není pouze v~lidské kultuře.  Se zvířaty
sdílíme mnoho instinktů, které se evolučně vyvinuly, disponujeme ale
širší škálou a~některé instinkty se také liší kvalitativně.

Cílem evoluční psychologie je identifikovat selekční tlaky, které
tvarovaly lidskou psýché v~průběhu evoluce, a~pak testovat, zda naše
psychologické mechanismy vůbec ukazují znaky, které by se daly
očekávat, pokud by byly vytvořeny k~řešení partikulárních adaptivních
problémů \citep{BarrettDunbarLycett2007}.

\subsection{Volba partnera v~evoluční psychologii}

Na lidského párování je zajímavé, že tento proces není možné popsat
jedinou strategií.  Jedna ze strategií spočívá v~dlouhodobém závazku,
který je často, ale ne vždy, charakteristický veřejným závazkem
(např. manželství).  V~dlou\-ho\-do\-bém párování oba partneři typicky
vkládají velké investice do potomka (nebo potomků).  Jako důsledek by
toto dle Batemanova principu a~z~něj odvozené Triversovy teorie
rodičovských investic mělo vést k~velké vybíravosti budoucího
partnera.  Špatná volba partnera by byla velmi nákladná buď pro muže
nebo ženu, protože riskují, že promarní své nákladné investice
\citep{Buss2007}.

Mezi další strategie párování patří krátkodobé párování, které může
trvat několik měsíců, několik týdnů, několik dnů nebo jen několik
minut.  Tato dočasnost vztahu silně ovlivňuje procesy výběru partnera
\citep{Buss2007}.

Lidé jeví pozoruhodné schopnosti tyto strategie míchat a~používat dle
potřeby.  Lidé nejsou ani čistě monogamní, ani čistě promiskuitní.
Nejsou ani polygamní, ale ani poly\-an\-dryn\-ní.  Kterou strategii si
dotyčný jedinec vybere záleží na situaci.  Volba je ovlivněna počtem
dostupných partnerů, vnímáním vlastní atraktivity nebo převládající
kulturní normou \citep{Buss2007,Murstein1970}.

\citet{BussEtAl1990,Buss2007} provedl velkou mezikulturní srovnávací
studii pre\-fe\-rencí 32 charakteristik vyhledávaných muži a~ženami pro
volbu potenciálního partnera pro dlouhodobý vztah.  Tato studie
srovnávala 37 různých kulturních prostředí na šesti kontinentech
a~pěti ostrovech.  Celková velikost vzorku byla $10047$, s~průměrem
272 pro každou ze 37 kultur.  Následující podsekce tedy pojednávají
pouze o~preferencích spojených s~volbou partnera pro dlouhodobý vztah.

\subsubsection{Kulturně ovlivněné vyhledávané vlastnosti}

Z~výsledků výzkumu vyplývá, že vlastností, která je nejvíce ovlivněna
kulturou, je cudnost (sexuální věrnost) a~panenství.  Mezi kultury,
které na tyto vlastnosti kladly enormní důraz, patří Čína, Taiwan,
Indie a~Írán.  Na opačné straně škály leží kultury, které na tyto
vlastnosti kladou pouze malý důraz -- Finsko, většina zemí Západní
Evropy (Dánsko, Švédsko, Francie a~Německo).  K~těmto kulturám lze
přiřadit i~dvě země Východní Evropy -- Bulharsko a~Jugoslávie,
a~některé africké kultury \citep{BussEtAl1990,Buss2007}.

\subsubsection{Mezi-kulturně univerzální vyhledávané vlastnosti}

Mnoho vlastností bylo shodně vyhledáváno oběma pohlavími.  Celosvětově
ženy i~muži vyhledávají partnery, kteří jsou inteligentní, milí,
chápající, spolehliví a~zdraví.  Vzájemná láska patřila mezi nejvíce
ceněné charakteristiky.  Obě pohlaví si také shodně cenila
potenciální partnery, se kterými se shodovali v~politických názorech
a~náboženském vyznání \citep{Buss2007}.

Na prvních místech žebříčku se z~povahových vlastností umístily u~obou
pohlaví následující proměnné: \emph{vzájemná láska}, \emph{vzdělání
  a~inteligence}, \emph{spolehlivost}, \emph{emocionální stabilita
  a~dospělost}, \emph{pří\-jem\-ná povaha} a~\emph{společenská
  povaha}.

\subsubsection{Univerzální rozdíly vyhledávaných vlastností mezi pohlavími}

Ženy a~muži na celém světě se lišili v~řadě charakteristik.  Tyto
pozorované rozdíly byly ve shodě s~předpoklady evoluční psychologie.

\paragraph{Získávání zdrojů}

Ženy signifikantně více než muži vyhledávaly ,,dobré finanční
vy\-hlídky``.  Ženy také měly tendenci vyhledávat kvality, které jsou
propojeny se získáváním zdrojů jako ambice, pracovitost, společenský
status a~vyšší věk muže \citep{Buss2007}.

Náklady na výchovu potomků díky neobvykle velkému mozku musí vynucovat
větší požadavky na muže, aby poskytovali rodičovskou péči ženě, se
kterou potomka počali.  Proto vzhledem k~těmto vysokým investicím do
výchovy potomka, budou ženy hledat partnera, který bude moci věnovat
prostředky na výchovu jejich společného potomka.  Pokud tento
předpoklad platí, tak budou ženy pravděpodobně preferovat muže, kteří
jsou schopni investovat do nich a~do společných potomků
\citep{Trivers1972,Ridley2007}.

\paragraph{Mladší věk}

Celkový počet potomků, které je schopna žena zplodit svému
poten\-cionál\-ní\-mu partnerovi, je přímo ovlivněn věkem.  Horní hranice
biologické plodnosti je u~každé ženy individuální, ale stále
neodvratitelná.  Mladistvost je důležitá s~ohledem na množství
potomků, protože se stoupajícím věkem je žena schopná muži
,,nabídnout`` méně a~méně potomků.  Jednou ze strategií, jak se
vyrovnat s~monogamií, je zvolit si za partnera, co nejmladší ženu,
která bude schopná za dobu svého reprodukčního cyklu zplodit mnoho
potomků.  Muži ve všech 37 kulturách vyhledávali mladší partnerky.
Tento konkrétní rozdíl mezi pohlavími byl nejsilnější v~celé studii
\citep{BussEtAl1990, Buss2007}.

\paragraph{Fyzická atraktivita}

Další rozdíly mezi pohlavími byly nalezeny v~poptávce fyzické
atraktivity \citep{Buss2007}.  Fyzická atraktivita je dána rysy
obličeje, výškou postavy, držením těla, hmotností, barvou vlasů
a~pokožky, věkem, poměrem pasu a~boků, u~žen velikostí a~tonusem
poprsí.  \citet{Slamenik2008} píše, že každá doba a~každé společenství
má svůj ideál krásy, který je sdílen většinou populace, ale kromě toho
má každý člověk svůj obraz fyzické atraktivity.  \citet{Buss2007}
k~tomu dodává, že standardy krásy nejsou voleny úplně libovolně a~ani
nejsou nekonečně kulturně variabilní.  Evoluční psychologie předkládá
teorii o~vývoji standardů ženské krásy -- důležité charakteristiky
a~vlastnosti ženské krásy jsou spojeny s~\emph{plodností} (okamžitou
schopností počít dítě) a~\emph{reprodukční hodnotou} (budoucí
reprodukční potenciál).

Ženská krása je postavena na triumvirátu mládí, postavy a~tváře
\citep{Ridley2007, BussSchmitt1993}.  Mužská krása je postavena na
stejném základě, ale ženy se rozhodují více dle osobnosti a~postavení
k~těmto indikátorům přidávají i~vzdělání \citep{Ridley2007,
  BussSchmitt1993, Buss2007}.  Krásné ženy se vdávají za bohaté muže
mnohem častěji, následně se dá usuzovat postavení muže z~krásy ženy.

K~dalším faktorům fyzické atraktivity patří poměr pasu a~boků.  Singh
(dle \citealp{BarrettDunbarLycett2007, Ridley2007}) formuloval
hypotézu, že muži vyhledávají partnerky dle proporce obvodu pasu vůči
obvodům hrudníku a~boků.  Poměr pasu k~bokům je pod kontrolou
pohlavních hormonů a~také je indikátorem celkového zdraví ženy
(srovnej s~teorií pohlavního výběru, kde pohlavní signály jsou znakem
imunitní zdatnosti a~vývojové stability). Poměr pasu a~boků je jednak
hodnocen jako indikátor fitness (biologické zdatnosti) a~jednak jako
indikátor vysoké schopnosti reprodukce.  Ideální tvar ,,přesýpací
hodiny`` je způsobený ,,gynoidním`` způsobem ukládání tuku -- více na
bocích, méně na trupu.

Jedním z~důsledků tohoto jevu je, že ženy ukazují (resp. ,,klamou``)
své mládí tím, že zdůrazňují neotení (mladistvé) prvky jako je malý
nos, malá brada, nebo neochlupená bledá pokožka \citep{Barber1995}.
Výsledky studie, kterou provedl Jones (cit dle
\citealp{BarrettDunbarLycett2007}), poukazují na to, že ženy, jejichž
odhadnutý věk byl nižší než jejich skutečný věk, byly považovány za
atraktivnější.

Další důležitým indikátorem pro volbu partnera je symetrie,
resp. fluktuační asymetrie. Především pak symetrie tváře
\citep{Ridley2007}.  Symetrie ukazuje na dobrou fitness, odolnost
proti nemocem a~parazitům.  Nicméně krása tváře není postavena jen na
symetrii. \citet{Ridley2007} k~tomu říká: ,,Dalším nápadným rysem
hezkého obličeje je, že se nám průměrné tváře líbí mnohem víc než
extrémy.`` Oba tyto rysy (,,symetrie obličeje``, ,,průměrnost``)
vysvětluje již zmíněná teorie pohlavního výběru, kde pohlavní signály
jsou znakem imunitní zdatnosti a~vývojové stability \citep{Barber1995}.

Evolučně-psychologické předpoklady týkající se fyzické atraktivity se
ve výsledcích multikulturního výzkumu potvrdily.  Ve všech kulturách
muži více než ženy vyhledávali atraktivitu nebo ,,pěkný vzhled``.
Důležité je zmínit, že ženy také u~potenciálních partnerů
vyhledávají fyzickou atraktivitu, ale v~nižší míře než muži.

\paragraph{Výška postavy muže}

Výška postavy muže se vymyká z principů fyzické atraktivity, o~kterých
jsme mluvili výše.  Ženy obecně preferují vyšší muže
\citep{Barber1995, Ridley2007}.  Vyšší muži jsou také vnímáni jako
společensky úspěšnější a~jsou také žádanější u~žen. Ženy vyhledávají
partnery, kteří jsou vyšší než ony \citep{Barber1995}.

\section{Předchozí výzkumy}
\citet{HarrisonSaeed1977} provedli obsahovou analýzu 800 seznamovacích
inzerátů, kde ověřovali hypotézu, že si lidé vybírají partnery na
podobné úrovni atraktivity. Analýzou bylo zjištěno, že ženy častěji
než muži nabízí fyzickou atraktivitu, vyhledávají finanční
zabezpečení, vyjadřují obavy o~motivech potenciálního partnera
a~vyhledávají staršího partnera. Komplementárně k~tomu muži častěji
než ženy vyhledávají fyzickou atraktivitu, nabízí finanční
zabezpečení, vyjadřují zájem k~ženitbě a~vyhledávají mladší
partnerky. Nabídka i~poptávka finančního zabezpečení se s~věkem
měnila. Poptávka/nabídka fy\-zické atraktivity a~charakterových
vlastnosti zůstávala s~věkem stabilní.  Ověřovaná hypotéza byla
potvrzena nízkou ale signifikantní korelací mezi celkovou úrovní
společenské žádoucnosti inzerenta a~celkovou úrovní společenské
žádoucnosti vyhledávaného partnera. Navíc bylo zjištěno, že dobře
vypadající inzerenti obou pohlaví vyhledávají dobře vypadající
partnery a~dobře vypadající ženy vyhledávají zabezpečené muže.

Další studii provedli \citet{GreenlessMcGrew1994}. Provedli obsahovou
analýzu inzerátů (N=1000) z~celonárodního magazínu, který vychází
každých 14 dní. Testovali celkem třináct hypotéz založených na
teoretickém základě evoluční biologie. Tato studie zjistila, že muži
více než ženy vyhledávají indície svědčící o~reprodukční hodnotě
(tj. fyzická atraktivita a~mládí). Ženy více než muži vyhledávají
náznaky, které odhalují schopnosti získávat zdroje (tj. aktuální
a~možné finanční zabezpečení a~starší věk muže). Ženy také
vyhledávaly ujištění o~mužově ochotě věnovat zdroje (ve formě času,
emocí, peněz a~statusu) v~partnerském vztahu. Obě dvě pohlaví nabízela
vlastnosti vyhledávané opačným pohlavím. Muži byli více promiskuitně
zaměřeni než ženy, preferovali více nezávazné partnerské vztahy a~byla
větší pravděpodobnost, že při podání inzerátu již (stále) budou
ženatí, zatímco ženy více vyhledávaly dlouhodobé monogamní partnerské
vztahy. Tyto rozdíly podporují evoluční předpoklady založené na
pohlavním výběru, rodičovských investicích a~reprodukčních
schopnostech. Také potvrzují použití seznamovacích inzerátů jako
vhodného výzkumného zdroje.

\citet{WayfordDunbar1995} provedli analýzu 881 seznamovacích inzerátů
ze čtyř ame\-ric\-kých novin, na kterých testovali preference pro volbu
partnera. Jejich výzkum potvrdil, že obecně muži preferují mladší
ženy, jejichž reprodukční hodnota je vysoká, a~ženy preferují muže,
kteří jsou lehce starší než ony samy. Dále ženy vyhledávají zdroje
a~muži vyhledávají fyzickou atraktivitu. Bylo také potvrzeno, že ženy
jsou vybíravější než muži. \citet{WayfordDunbar1995} dále testovali
předpoklady založené na hypotéze, že individuální preference budou
souviset s~tím, co jedinec může nabídnout. Ukázali, že:

\begin{itemize}
\item ženy se stávají méně náročnými se zvyšujícím se věkem
  (pravděpodobně protože jejich reprodukční hodnota s~rostoucím věkem
  klesá)
\item muži se s~věkem stávají více náročnými (pravděpodobně protože
  oplývají více zdroji)
\item ženy, které nabízí fyzickou atraktivitu mají vyšší požadavky než
  ty, které fyzickou atraktivitu nenabízí
\item muži nabízející zdroje mají vyšší požadavky než ti, kteří zdroje
  nenabízí
\item muži s~méně zdroji se snaží tuto nevýhodu vyvážit nabídkou
  rodinného závazku
\item muži a~ženy, kteří mají v~péči potomka, mají nižší nároky než ti
  bez potomků
\item jedinci z~vyšších socioekonomických tříd (kteří pravděpodobně
  oplývají více zdroji) mají vyšší požadavky než jedinci z~nižších
  tříd
\end{itemize}

\citet{Gil-BurmanPelaezSanchez2002} analyzovali vyhledávané a~nabízené
vlastnosti z~různých věkových skupin na vzorku 7415 seznamovacích
inzerátů ve španělských novinách. Nálezy tohoto výzkumu týkající se
věku, socioekonomického statutu a~fyzické atraktivity podporují
evoluční předpoklady o~výběru partnera. Výzkum však zaznamenal změnu
v~pre\-fe\-ren\-cích mezi ženami pod čtyřicet let. Autoři výzkumu toto
pokládají za výsledek so\-cio\-eko\-no\-mické transformace Španělska. Ženy pod
čtyřicet let u~mužů především vyhledávají fyzickou atraktivitu a~ženy
nad čtyřicet let vyhledávají hlavně socioekonomický status. Nejvíce
vyhledávanou vlastností muži všech věků je fyzická atraktivita.
Vlastnosti vyhledávané a~nabízené inzerenty mohly být ovlivněny osobní
situací inzerenta. Autoři studie dále uvádějí domněnku založenou na
průměrném věku inzerentů (okolo 40 let) a~společenských indikátorech
ve Španělsku, že většina inzerentů nebyla úspěšná ve výběru partnera
v~běžném věku pro vytvoření partnerského vztahu.

\citet{PawlowskiKoziel2002} se zabývali dopady inzerovaných
vlastností v~poměru k~zaslaným odpovědím na inzeráty.  Analýzou
odpovědí na 551 inzerátů podaných muži a~617 podaných ženami
v~lokálních polských novinách odvozovali ,,úspěšnost zásahu`` (počet
odpovědí). Analýzu provedli pomocí metody GLM (Generalize Linear
Model), kde počet odpovědí byla závisle proměnná a~vlastnosti nabízené
v~inzerátech jako věk, nejvyšší dosažené vzdělání, místo pobytu,
manželský stav, výška, váha, nabízené zdroje a~atraktivita byly
nezávisle proměnné. Vlastnosti, které ovlivňovaly úspěšnost inzerátů
u~můžu byly (pořadí dle důležitosti): nejvyšší dosažené vzdělání, věk,
výška a~nabízené zdroje. Tyto proměnné všechny pozitivně korelovaly
s~počtem odpovědí na inzerát. Naopak u~žen některé vlastnosti
negativně korelovaly s~počtem odpovědí: váha, výška, nejvyšší dosažené
vzdělání a~věk. Zdroje nabízené muži měly jen malý pozitivní efekt
a~inzerování atraktivity ženami nemělo na počet odpovědí efekt žádný,
což naznačuje, že lidé odpovídající na inzeráty se raději spoléhají na
více objektivní vlastnosti, jako je dosažené vzdělání, mužská výška
a~ženská váha, než na vlastnosti, které mohou být více ovlivněny
subjektivním vnímáním nebo záměrnou manipulací.

\Citet{CamposOttaSiquera2002} provedli analýzu seznamovacích inzerátů
publikovaných v~brazilských novinách.  Výzkum se zaměřil na obsah
inzerátů s~ohledem na nabízené a~poptávané vlastnosti, ale také na
prognózu počtu odpovědí na jednotlivé inzeráty.  Podávání inzerátu
v~novinách bylo zpoplatněné, stejně jako odpovědi na inzeráty, což
mohlo působit jako vstupní bariéra pro inzerenty i~respondenty.
Celkový počet analyzovaných inzerátů byl 807 (411 žen a~396 mužů).
Výsledky studie ukázaly, že požadavky na potenciálního partnera se
měnily s~věkem: požadavky žen s~rostoucím věkem klesaly a~naopak
požadavky mužů s~rostoucím věkem stoupaly.  Počet odpovědí na inzeráty
byly ovlivněny věkem podobným způsobem.  Starší ženy dostaly méně
odpovědí než ženy mladší a~starší muži dostali více odpovědí na
seznamovací inzeráty než muži mladší.

\section{Online seznamování}

Termín 'Online seznamování' je zde používán pro způsob seznámení pomocí
internetových serverů, kde je možné si podat seznamovací inzerát
a~ostatní uživatelé tohoto serveru mohou přes internet na tento
inzerát odpovědět \citep{HallEtAl2010,EllisonEtAl2006}.

Původně byl tento druh seznamovaní jistým druhem stigmatu, v~poslední
době se tento druh seznamování přesouvá do mainstreamu díky vysoké
penetraci internetu mezi uživateli \citep{GibbsEtAl2006, HallEtAl2010,
  EllisonEtAl2006}.

V~porovnání s~klasickým zprostředkovaným seznamováním, které bylo
zkoumáno dříve, má výzkum online seznamování výhodu v~mnohem širší
uživatelské základně.  Seznamování přes internet dává inzerentům také
více možností, jak se prezentovat pozitivně a~uvážlivě
\citep{GibbsEtAl2006}.

Uživatelé online seznamování vykazují jistou míru lhaní.  Muži
vykazují obecně vyšší míru náchylnosti ke~klamání včetně osobních
zájmů a~povahových vlastností, ženy více nepodávaly přesnou informaci
o~své váze \citep{HallEtAl2010}.

\citet{HallEtAl2010} uvádí, že lhaní v~seznamovacích inzerátech nemusí
být unikátní pro CMC.  Inzerenty může od přehnaného vylepšování
vlastního popisu odradit předpoklad setkání ,,tváří v~tvář`` (FtF --
Face to Face).  Nicméně i~tento výzkum potvrdil, že ke klamu dochází
v~oblastech, které jsou důležité pro evolučně-psychologickou volbu
partnera.  Muži více než ženy klamou o~finančním zajištění a~o~svém
věku, ženy více klamou o~skutečné váze.

\chapter{Výzkumná část}
\section{Cíl výzkumu a~formulace hypotéz}
Cílem výzkumné části této práce je na základě kvantitativní analýzy
seznamovacích inzerátů potvrdit hypotézy, které vyplývají z~evoluční
psychologie.

Dle předchozích výzkumů \citep{BussEtAl1990,GreenlessMcGrew1994,%
  Gil-BurmanPelaezSanchez2002,WayfordDunbar1995,HarrisonSaeed1977}
muži na ženách více vyhledávají fyzickou atraktivitu a~mládí, tedy
parametry zaručující vyšší plodnost, a~ženy vyhledávají u~mužů zdroje,
které jim pomohou vychovat potomka.

Výzkumná otázky tedy zní: ,,Jsou principy výběru partnera z~pohledu
evoluční psychologie promítnuty do online seznamovacích inzerátů$?$``

\paragraph{Biologické faktory pro volbu partnera}

Kulturně univerzální rozdíly mezi vy\-hle\-dá\-vanými vlastnostmi zahrnují
znaky, které u~žen indikují plodnost a~reprodukční hodnotu.  Prvním
z~těchto indikátorů je fyzická atraktivita.

\emph{H1a: Muži více než ženy budou v~inzerátech více vyhledávat parametry fyzické
  atraktivity.}

\emph{H1b: Ženy více než muži budou v~inzerátech nabízet a~popisovat svůj
  fyzický vzhled.}

Muži také vyhledávají mladší ženy.  Mladší věk je spojen s~vyšší
reprodukční hodnotou.  Čistě biologicky jsou mladší ženy schopny počít
muži více potomků, a~proto je jejich reprodukční hodnota vyšší.

Ženy na druhé straně používají vyšší věk muže jako indikátor spojený
se zajišťováním zdrojů.

\emph{H2a: Muži více než ženy budou v~inzerátech více vyhledávat mladší věk.}

\emph{H2b: Ženy více než muži budou v~inzerátech častěji vyhledávat starší věk.}

Jediná charakteristika, která se vymyká standardnímu schématu
vyhledávaní fyzických charakteristik, je výška muže (ženy vyhledávají
vyšší muže).

\emph{H3a: Muži více než ženy budou v~inzerátech uvádět, že jsou vysocí.}

\emph{H3b: Ženy více než muži budou v~inzerátech častěji poptávat vysoké muže.}

\paragraph{Volba partnera dle investicí do výchovy potomka}

Ženy u~mužů při seznamování především vyhledávají zajištění a~závazek,
aby byla zajištěna výchova společného potomka, a~tento princip se bude
promítat i~do seznamovacích inzerátů.

\emph{H4a: Ženy více než muži budou v~inzerátech požadovat zdroje/prostředky
  reprezentované majetkem nebo statusem (např. zajištěný).}

\emph{H4b: Muži více než ženy budou v~inzerátech nabízet parametry
  dokládající jejich možnosti poskytnout partnerovi zdroje/prostředky
  ve formě majetku nebo statusu.}

\emph{H5a: Ženy více než muži budou v~inzerátech vyhledávat závazek.}

\emph{H5b: Muži více než ženy budou v~inzerátech nabízet závazek.}

\paragraph{Smysl pro humor/sociální schopnosti}

Mezi nejvyhledávanější vlastnosti potenciál\-ního partnera patří
vlastnosti, které \citet{BarrettDunbarLycett2007} shrnují do kategorie
\emph{Sociální a~interpersonální schopnosti}, a~vlastnosti, které
stejní autoři umísťují do kategorie \emph{Závazek}.

\emph{H6a: Sociální a~interpersonálních schopnosti budou patřit
  k~nejvyhledávanějším vlastnostem u~obou pohlaví.}

\emph{H6b: Závazek bude patřit k~nejvyhledávanějším vlastnostem u~obou
  pohlaví.}

\section{Výzkumný soubor}
Výzkumný soubor byl vybrán z Internetových seznamek provozovaných
v~českém jazyce, kterých je poměrně velký počet, a~proto bylo nutné
vybrat nejvíce relevantní zdroje.  Portál Seznam.cz uvádí počet
vyhledaných odkazů na klíčové slovo ,,seznamka`` přes 7 miliónů,
vy\-hle\-dávací stroj Google.cz vrátil na dotaz ,,seznamka`` přes 3,5
miliónu odkazů.  Pro účely tohoto výzkumu byly identifikováno deset
nejvíce relevantních seznamek, které se ve výsledcích vyhledávání
Seznam.cz a~Google.cz objevují na prvních pozicích: seznamka.cz,
seznamka.lide.cz, stesti.cz, znamost.cz, seznamka.tiscali.cz,
rande.cz, seznamka.deni.cz, seznamit.cz a~najdisime.cz.  Z~tohoto
seznamu byly vynechány seznamky zaměřené pouze na seznámení za účelem
erotiky či sexu, protože se tento výzkum zabývá seznamováním za účelem
vytvoření dlouhodobého vztahu.  Parametry výběru partnera ,,na jednu
noc`` se budou lišit od parametrů výběru partnera pro dlouhodobý vztah
\citep{Buss2007}.  Z~těchto deseti seznamek byly vyřazeny seznamky,
které uživatele příliš vedly formou strukturovaného formuláře,
a~seznamky, kde bylo nutné se zaregistrovat pro zobrazení více
informací.  Ze zbylých seznamek byly vybrány tři s~nejvyšším počtem
seznamovacích inzerátů: seznamka.lide.cz, seznamka.cz a~rande.cz.
Tyto tři seznamky také mají podobnou strukturu inzerátů, obsahují
volný text inzerátu a~ve strukturovaných datech věk, lokalitu
a~přezdívku inzerenta.  Po důkladnější analýze struktury těchto online
seznamek se ukázalo, že seznamky seznamka.cz a~rande.cz mají stejného
provozovatele a~také stejný zdroj dat pro inzeráty.  Protože byly
inzeráty v~těchto dvou seznamkách stejné, byl do výzkumu zařazen pouze
jeden z~těchto serverů -- seznamka.cz.

Inzeráty v~těchto online seznamkách jsou umístěny v~kategoriích. Tyto
kategorie jsou rozděleny buď dle typu hledaného vztahu (vážný, kamarádský,
dopisování, atp.) nebo dle hledaného pohlaví inzerenta a~požadovaného
partnera (On hledá ji, Ona hledá ji).

Inzeráty v~těchto seznamkách je možné zadávat zdarma, seznamka.cz také
umožňují zvýhodnění inzerátu na předních pozicích pomocí mikroplateb
pomocí mobilního telefonu.

Výzkumný soubor zahrnuje inzeráty z~kategorie ,,Vážná seznámení`` ze
seznamovacích serverů seznamka.cz a~seznamka.lide.cz za období
1.4.2009 -- 31.3.2010.  Výběr in\-ze\-rá\-tů byl dále omezen na
heterosexuální respondenty (podkategorie ,,On hledá ji`` a~,,Ona hledá
jeho``) a~byly vybrány pouze inzerenti věku mladé dospělosti (20 až 40
let včetně).  Celkový počet inzerátů za vybrané období po vyřazení
duplicit a~inzerátů ze zahraničí na seznamovacím serveru
seznamka.lide.cz byl 12650 a~na seznamovacím serveru seznamka.cz
17933.  Z~těchto inzerátů bylo náhodným výběrem vybráno celkem 1830
inzerátů (880 ,,On hledá ji``, 950 ,,Ona hledá jeho``).  Pro náhodný
výběr byl použit algoritmus \citet{WichmannHill1982} implementovaný ve
standardní knihovně programovacího jazyka Python.

\section{Použité metody}
Výzkum probíhal ve dvou fázích. V~první fázi bylo vybráno celkem 400
inzerátů (131 muži, 101 ženy, 69 inzerátů mužů vyřazeno, 99 inzerátů
žen vyřazeno).  Polovina těchto seznamovacích inzerátů byla vybrána na
serveru seznamka.cz a~druhá polovina na serveru seznamka.lide.cz.
Těchto 400 inzerátů bylo určeno na seznámení se s~kódovací technikou
a~její nácvik.  Na základě vyřazovacích kritérií uvedených níže bylo
v~pilotáži vyřazeno 34,5\% inzerátů ,,On hledá ji`` a~49,5\% inzerátů
,,Ona hledá jeho``. Tyto poměry byly stabilní i~při přepočtu na
jednotlivé servery. Na základě těchto poměrů bylo do hlavního
výzkumného vzorku náhodným výběrem zvoleno 680 inzerátů ,,On hledá
ji`` a~750 inzerátů ,,Ona hledá jeho``, tak aby výsledný počet
inzerátů po vyřazení nevyhovujících odpovídal přibližně počtu 1000
inzerátů. Z~hlavního vzorku byly před výběrem inzerátů vyřazeny ty
inzeráty, které byly použity pro pilotní průzkum.

Na základě pilotáže byly také upraveny podmínky pro výběr
inzerátů. V~prvním fázi byly kódovány inzeráty ze serveru
seznamka.lide.cz. Ve druhé fázi pilotáže byly vyřazeny inzeráty
cizojazyčné a~ze zahraničí. Z~obsahu každého inzerátu byl vypočten
unikátní identifikátor pomocí hašovací funkce SHA-1, ze kterého byly
použito prvních 8 znaků (ex\-pe\-ri\-mentálně bylo ověřeno, že od 8 znaků
výše již ve výsledných unikátních identifikátorech nejsou žádné
kolize). Tento unikátní identifikátor byl použit jednak na vyřazení
duplicit a~jednak na vyřazení inzerátů použitých v~pilotáži z~dalšího
výzkumu.

Hlavní výzkumný vzorek čítal 1430 inzerátů (680 ,,On hledá ji``, 750
,,Ona hledá jeho``), který byl vybrán náhodným výběrem z~všech
inzerátů, ze kterých byly automaticky vyřazeny:

\begin{enumerate}
\item inzeráty, které se opakují (mají stejné znění)
\item inzeráty, které byly použity v~pilotním průzkumu
\item inzeráty, které neměly jako lokalitu uvedenou českou republiku
\end{enumerate}
Při kódovacím procesu byly následně vyřazeny tyto typy inzerátů:

\begin{enumerate}
\item inzeráty, které byly umístěny ve špatné kategorii.
\item inzeráty, které zjevně nebyly podány za účelem vážného seznámení
  nebo byly podány za účelem pouze sexuálního styku.
\item inzeráty obsahující žádnou či nejasnou poptávku či nabídku
  inzerenta.
\item inzeráty, které byly podány více než jednou osobou, protože
  takové inzeráty nereflektují individuální preference volby partnera.
\item inzeráty, ze kterých vyplývá, že inzerent již má děti. Inzerent
  s~dětmi může mít změněny preference volby partnera, protože může
  hledat zdroj zabezpečení pro stávající potomky
  \citep{WayfordDunbar1995}.  Nebyly vyřazeny inzeráty, které
  obsahovaly pouze implicitní náznak toho, že respondent má děti,
  protože z~poptávky ,,mít rád děti``, nebylo zřejmé, zda-li
  inzerent/-ka děti už má nebo by je jen chtěl/a mít.  Také je možné,
  že někteří jedinci v~inzerátech děti úplně zatajili, protože dítě
  může být v~seznamování handicapem.  Muži musí dělit své zdroje na
  péči o~více dětí (a~to i~když s~nimi děti nežijí) a~ženy musí péči
  dělit mezi více dětí a~pravděpodobně budou požadovat zdroje i~pro
  dítě, které s~budoucím partnerem nebude geneticky příbuzné. Také je
  možné, že již další děti nebude chtít.
\item inzeráty ze zahraničí, nebo psané cizím jazykem.  Odlišné
  sociokulturní a~jazykové prostředí by mohlo ovlivnit preference
  volby partnera.
\item inzeráty osob se zdravotním postižením byly z~výběru také
  vyřazeny, protože jejich preference volby partnera budou
  pravděpodobně ovlivněné postižením, např. mohou mít jiné nároky na
  potenciálního partnera.
\item inzeráty, kde byl uveden rozdílný vyšší nebo nižší věk než byla
  vybrané rozmezí mladší dospělosti (20 až 40 let).
\item Ostatní nestandardní inzeráty.  Příklad: ,,Hledam fanouska
  Michaela Jacksona, ktery ho posloucha a ktery mu je i trosku
  podobny. Hledam trvaly vztah. Foto zaslu po odpovedi``.
\end{enumerate}

Po vyřazení inzerátů na základě těchto pravidel vznikl výzkumný vzorek
o~velikosti N=723 (371 mužů, 352 žen).

Každý inzerát obsahoval informaci o~věku a~pohlaví inzerenta,
lokality, ve které se nachází, s~přesností na kraje + hlavní město
Praha, a~dichotomickou proměnou o~přítomnosti fotografie. Seznamka.cz
lokalitu neuvádí jako kraj, ale přímo město. Z~důvodů unifikace
inzerátů se serverem seznamka.lide.cz došlo k~překódování jednotlivých
měst na kraje.

Stažené inzeráty byly pomocí počítačových programů grep a~sed
transformovány do CSV souborů, kde jako oddělovač byl použit znak
,,{\textbar}``, který se v~běžném textu nevyskytuje. Jediný inzerát,
který tento znak obsahoval byl ručně opraven nahrazením za znak
lomítka. Byly vytvořeny celkem čtyři soubory -- dva x dva pro
seznamovací server a~pohlaví. Do těchto souborů bylo pomocí skriptu
add\_hash.py (v~příloze) přidán unikátní identifikátor SHA-1 (viz
výše). V~dalším kroku byly pomocí skriptu get\_random.py (v~pří\-loze)
vybrány náhodné inzeráty pro pilotáž a~tyto inzeráty byly následně
z~hlavních datových souborů odstraněny. Po dokončení pilotáže byl
pomocí skriptu get\_random.py vybrán hlavní výzkumný vzorek, který byl
uložen opět jako soubor CSV.  Tentokrát již byly použity pouze soubory
dva -- kódování probíhalo samostatně pro muže a~pro ženy.  Tyto
soubory byly naimportovány do tabulkového procesoru OpenCalc
z~kancelářského balíku OpenOffice.org. Kódování bylo prováděno
dvoufázově. V~první fázi byly kódovány pouze klíčová slova do dvou
samostatných sloupců s~poptávkou a~nabídkou. Výsledek tohoto prvního
kola byl vyexportován opět do formátu CSV a~následně zpracován pomocí
skriptu code.py (v~příloze), který automaticky překódoval použitá
klíčová slova do vybraných kategorií (závazek, atraktivita,
status/majetek, atd.).  Výsledná kontingenční tabulka byla uložena do
jednoho výsledného souboru CSV, který byl naimportován do programu
SPSS 18.0.

Pro zjištění signifikantních rozdílů mezi muži a~ženami byl použit
chí-kvadrát test nezávislosti, který je vhodným nástrojem pro
zjišťování rozdílů mezi skupinami v~případě nominálních proměnných.

\section{Kódování dat}
V~souladu s~předchozími výzkumy
\citep{HarrisonSaeed1977,GreenlessMcGrew1994,BarrettDunbarLycett2007}
byly kromě věku, pohlaví, lokality a~indikátoru přítomnosti fotografie
zkoumány ještě další proměnné, které byly separátně kódovány na
charakteristiky nabízené, které zvyšují sociální žádoucnost inzerenta,
a~charakteristiky požadované od potenciálního partnera. Takovýchto
proměnných bylo celkem 5: ,,fyzická atraktivita``, ,,majetek/status``,
,,závazek``, ,,sociální schopnosti``, ,,koníčky/zájmy``. Kódování
požadovaného a~nabízeného vzdělání bylo zavrženo na základě příliš
malého počtu inzerátů, ve kterých se tato proměnná
objevovala. Vzdělaní je zohledněno v~proměnné ,,majetek/sta\-tus`` a~to
pouze v~případě vysokoškolského vzdělání. Těchto pět proměnných bylo
kódováno frekvenčně na počet vyskytujících se klíčových slov.

Kromě těchto proměnných nabídky a~poptávky, byly kódovány ještě 3
další proměnné: ,,požadavek na fotografii v~odpovědi na inzerát``,
,,požadavek na vyšší věk``, ,,požadavek na nižší věk``. Všechny tyto
proměnné byly kódovány pouze v~případě pozitivní zmínky v~inzerátu
(tj. např. fráze ,,sport NE`` nebyla kódována). Požadované nebo
nabízené charakteristiky, které byly ,,přiměřené`` nebo ,,stejné``
taktéž nebyly kó\-do\-vá\-ny, protože nelze spolehlivě odhadnout, co
se pod pojmem doopravdy skrývá (např. požadavek na ,,přiměřený věk``
muže).

Kódovací mechanismus do kategorií vzniklý na základě kompilace předchozích výzkumů
a~pilotáže je následující:

\begin{enumerate}
\item \emph{Fyzická atraktivita.} Pozitivní skóre v~této kategorii
  bylo přiřazeno v~případě, že inzerent o~sobě prohlašuje, že dobře
  vypadá, nebo požaduje dobře vypadajícího partnera. Pozitivní skóre
  bylo přiřazeno v~případě například těchto klíčových slov:
  \emph{atletická postava, atraktivní, dlouhonohá, dobře stavěný,
    drobná, hezký/á, hezké postavy, hubený, charismatický/á, krásná,
    malá, menší, mladistvá, mladý, pěkný/á, pěkných tvarů, pohledný/á,
    přitažlivý/á, roztomilý/á, sex-appeal, smyslná, sportovní,
    sportovní postava, statný, svalnatý, šarmantní, štíhlý/á, udělaný,
    upravený/á, u\-rost\-lý, velký, vlasatý, voňavý, vysoký/á, zajímavě
    vypadající, zdravý/á, ženských tvarů}...  Dále byly do této
  kategorie přiřazeny i~klíčová slova týkající se barvy vlasů a~očí:
  \emph{brunetka, blondýnka, modrooká, hnědé oči, vlasatý} atp.

\begin{enumerate}
\item \emph{Výška.}. Parametry týkající se výšky byly kódovány samostatně.
  Klíčová slova zahrnují: \emph{vysoký/á, vyšší}. V~případě, že byla výška
  uvedena číslem, byla do této kategorie kódována v~případě, že byla
  větší než 180 cm, což byla průměrná výška můžu ve věku 18--24 let
  v~roce 2002 \citep{UZIS2002}.
\end{enumerate}
\item \emph{Status/Majetek.} Pozitivní skóre v~této kategorii bylo přiřazeno
  v~případě, že inzerent nabízí nebo hledá příjem, majetek nebo jiné
  peněžní či nepeněžní zdroje. Klíčová slova zahrnují:
  \emph{ambiciózní, bohatý/á, finančně nezávislý/á, finančně
    zabezpečený/á, finančně zajištěný/á, finančně zajištěný, golf,
    chytrá, inteligence, inteligentní, peníze, perspektivní,
    podnikání, podnikatel/ka, podnikavý, práce, pracovitý, pracující,
    příjem, s~autem, s~bytem, s~domem, samostatný/á, schopný/á,
    soběstačný/á, student/ka VŠ, studující, štědrý, úspěšný/á,
    vysokoškolské vzdělání, vyšší životní standard, vzdě\-la\-ný/á,
    zabezpečený/á, zajištěná/ý, zaměstnán/a, odborník/ice (např.
    lékař, právník)}, termíny předpokládající vyšší životní standard
  (\emph{kvalitní parfémy, zahraniční dovolená, drahá auta, exklusivní
    lokalita bydliště, dobré víno, móda}).
\item \emph{Závazek.} Pozitivní skóre v~této kategorii bylo přiřazeno v
  případě, pokud inzerent vyjadřuje zájem o~vážný vztah nebo
  nabízí/hledá nemateriální zabezpečení. Tato ka\-te\-gorie je na rozdíl
  od výzkumu Harrisona a~Saeedové \citeyearpar{HarrisonSaeed1977} také kódována odděleně do
  tříd poptávka a~nabídka. Klíčová slova spojena s~touto kategorií
  zahrnují: \emph{péče o~potomka, sdílení, mít rád/a, vážit si, opora,
    podpora, ochrana, láska, pocit bezpečí. milý/á, (emocionálně)
    stabilní, zralý/á, zodpovědný/á, umí potěšit, dobrý/á ku\-chař/ka,
    pečující, orientovaný/á na rodinu, hodný/á, pohodový/á, vážný/á,
    citlivý/á, schopný/á naslouchat, něžný/á, soucitný/á, chápající,
    vážný vztah, vážné seznámení, manželství, svatba, vyjádření zájmu
    mít v~budoucnu děti, chápavý/á}.

\item \emph{Povahové vlastnosti.} Tato kategorie byla na základě pilotáže,
  kde se některé povahové vlastnosti objevovaly častěji, rozdělena do
  následujících podkategorií:

\begin{enumerate}
\item \emph{Sociální a~interpersonální schopnosti.} Výsledky pilotáže
  ukázaly, že smysl pro humor, vtipnost, apod. je jednou z~nejčastěji
  žádaných povahových vlastností, což je v~souladu s~výsledky
  multikulturní studie \citep{BussEtAl1990}.  Z~těchto důvodů byla
  tyto klíčová slova vyčleněna jako samostatná podkategorie Sociální
  a~interpersonální schopnosti. Klíčová slova zahrnují: \emph{smysl
    pro humor, vtipný/á, zábavný/á, živý/á, hodně se směje, veselý/á,
    společenský/á, pohodový/á, pří\-jemný/á, sympatický/á, posezení
    s~přáteli}.
\item \emph{Věrnost/Upřímnost.}  Klíčová slova zahrnují: \emph{věrný/á, věrnost,
  upřímnost, upřimný/á, čestný/á, poctivý/poctivá}.
\item \emph{Ostatní.} Tato kategorie zahrnuje všechny ostatní povahové
  vlastnosti. Klíčová slova zahrnují: \emph{romantik, romantický/á,
  tolerantní, charakterní, sebevědomý/á, zajímavý/á, kompromisní}.
\end{enumerate}
\item \emph{Koníčky a~zájmy.} Koníčky a~zájmy jsou roztříděny do podkategorií
  podle britského vzoru \citep{GreenlessMcGrew1994}:

\begin{enumerate}
\item \emph{Nákladné.} Do této kategorie byly zařazeny koníčky, které jsou
  finančně nákladné: \emph{cestování, auta, jachting, golf,
    motorismus, motorky}.
\item \emph{Statusové.} Do této kategorie byly zařazeny koníčky spojené
  s~určitým spole\-čen\-ským statusem, vytříbeností nebo sofistikovaností:
  \emph{opera, balet, divadlo, umění}.
\item \emph{Fyzické.} V~této kategorii jsou zařazeny všechny koníčky
  a~záliby, které vyžadují fyzickou aktivitu: \emph{sport, tanec,
    výlety, procházky, příroda, aktivní, plesy, turistika, hory, kolo,
    plavání, další sporty, vycházky}.
\item \emph{Ostatní.} Do této kategorie byly zařazeny všechny ostatní koníčky
  a~záliby: \emph{cizí jazyky, kino, čtení, vaření, chalupaření,
    hudba, knihy, zvířata, filmy, fotografování}.
\end{enumerate}
\item \emph{Hledá staršího/mladšího partnera.}  Pozitivní skóre v~těchto
  kategoriích bylo přiřazeno v~případě, že v~inzerátu bylo explicitně
  vyjádřeno, že inzerent hledá staršího, resp. mladšího partnera.
  V~případě věkového rozmezí musel být střed požadovaného intervalu
  věkového rozmezí větší, resp. menší, než věk inzerenta.  Tedy pokud
  inzerentka uvedla vlastní věk 30 let a~požadovala věk
  potenciálního partnera 29--40, tak tento inzerát byl kódován do
  kategorie ,,Starší věk``, protože průměr požadovaného rozmezí je
  35,5, což je více než věk inzerentky.
\item \emph{Požaduje fotografii.} Pozitivním skóre v~této kategorii byly
  kódovány inzeráty, ve kterých inzerent vyjádřil, že vyžaduje
  v~odpovědi na inzerát fotografii.  Nabídka fotografie byla pro
  zjednodušení překódována do proměnné ,,přítomnost fotografie``.
\end{enumerate}
\section{Výsledky}

\ctable[
  cap = Poptávka a~nabídka fyzické atraktivity,
  caption = Procentuální rozložení vyhledávaných a~nabízených parametrů fy\-zické atraktivity u~mužů a~žen,
  label   = tab_atr,
  center,
  pos = htbp,
  botcap,
  notespar
]{lrrrrr}{
  \tnote[*]{$p < 0.05$}
  \tnote[**]{$p < 0.01$}
  \tnote[***]{$p < 0.001$}
}{                                                                       \FL
  & \multicolumn{2}{c}{Muži (N=371)} & \multicolumn{2}{c}{Ženy (N=352)} & \NN
  \multicolumn{1}{c}{(N=723)} %
                 &   N &     \% &   N &     \% & Pearsonův $\chi^2$ \ML
  \emph{Fyzická atraktivita} & & & & & \NN
  ~~Poptávaná    & 101 & 27,2\% &  40 & 11,3\% & 29,048\tmark[***] \NN
  ~~Nabízená     &  72 & 19,4\% &  87 & 24,8\% &  6,238\tmark[*]~~   \ML
  \emph{Výška} &     &        &     &        & \NN
  ~~Poptávaná    &   3 &  0,8\% &  29 &  8,2\% & 21,849\tmark[***] \NN
  ~~Nabízená     &  22 &  5,9\% &   5 &  1,4\% &  9,001\tmark[**]~  \ML
  \emph{Fotografie} & & & & & \NN
  ~~Přítomná     & 135 & 36,4\% & 69 & 19,6\% & 24,305\tmark[***] \NN
  ~~Poptávaná    & 17 & 4,6\% & 45 & 12,8\% & 14,470\tmark[***] \LL
}

Tabulka~\ref{tab_atr} ukazuje, že muži vyhledávají parametry fyzické
atraktivity více než ženy ($\chi^2$~=~29,048, df~=~2,
p~{\textless}~0.001), což potvrzuje hypotézu H1a. Nabídka v~tomto
případě také reflektuje poptávku a~ženy častěji než muži
($\chi^2$~=~6,184, df~=~2, p~{\textless}~0.05) uvádějí v~inzerátech
parametry fyzické atraktivity, což potvrzuje hypotézu H1b. Nicméně
síla asociace je u nabídky (Cramerovo~V~=~0,92) nižší než u poptávky
(Cramerovo~V~=~0,2).

Z~tabulky~\ref{tab_atr} lze také vyčíst, že ač ženám méně záleží na
fyzické atraktivitě, tak toto neplatí pro výšku, ženy výšku častěji
poptávají ($\chi^2$~=~231,849, df~=~1, p~{\textless}~0,001) a~také
muži výšku častěji nabízejí ($\chi^2$~=~9,001, df~=~1,
p~{\textless}~0,005).  Tyto výsledky potvrzují hypotézy H2a a H2b.

Do tabulky~\ref{tab_atr} byla také zařazena proměnnou ,,fotografie``,
která také souvisí s~fy\-zickou atraktivitou, resp. vzhledem. Výsledky
u~této proměnné jsou opačné vůči vy\-hle\-dávaný a~nabízeným klíčovým
slovům fyzické atraktivity.  Ženy poptávají fotografii častěji než
muži ($\chi^2$~=~14,470, df~=~1, p~{\textless}~0,001) a~muži častěji
než ženy mají u~inzerátu zobrazenu fotografii ($\chi^2$~=~25,127,
df~=~1, p~{\textless}~0,001).

\ctable[
  cap = Poptávka a~nabídka věkového rozdílu,
  caption = Procentuální rozložení vyhledávaného věkového rozdílu mezi muži a~ženami.,
  label = tab_age,
  center,
  pos = htbp,
  botcap,
  notespar
]{lrrrrr}{
  \tnote[*]{$p < 0.05$}
  \tnote[**]{$p < 0.01$}
  \tnote[***]{$p < 0.001$}
}{                                                                        \FL
  & \multicolumn{2}{c}{Muži (N=371)} & \multicolumn{2}{c}{Ženy (N=352)} & \NN
  \multicolumn{1}{c}{(N=723)} %
                 &   N &     \% &   N &     \% & Pearsonův $\chi^2$       \ML
  \emph{Starší věk} & & & & &                                             \NN
  ~~Poptávaný    &   0 &     -- &  92 & 26,1\% & 108,763\tmark[***]       \ML
  \emph{Mladší věk} & & & & &                                             \NN
  ~~Poptávaný    &  66 & 17,8\% &   1 &  0,3\% &  63,766\tmark[***]       \LL
}

V~souladu s~předchozími výzkumy \citep{WayfordDunbar1995} lze
v~Tabulce \ref{tab_age} vidět, že muži více než ženy vyhledávají
mladší věk ženy ($\chi^2 $~=~63,766, df=1, p~{\textless}~0,001), mezi
kódovanými inzeráty byl pouze jeden inzerát ženy, která hledala
mladšího partnera, a~také, že ženy vyhledávají staršího partnera více
než muži ($\chi^2 $~=~108,763, df~=~1, p~{\textless}~0,001), resp. ve
zvoleném vzorku vážných seznámení nebyl ani jeden inzerát, který by
vyhledával starší ženu. Tímto jsou potvrzeny hypotézy H3a a~H3b.

\ctable[
  cap = Poptávka a~nabídka statusu nebo majetku,
  caption = Procentuální rozložení inzerátů nabízejících nebo poptávajících parametry indikující zdroje ve formě statusu nebo majetku,
  label   = tab_status,
  center,
  pos = htbp,
  botcap,
  notespar
]{lrrrrr}{
  \tnote[*]{$p < 0.05$}
  \tnote[**]{$p < 0.01$}
  \tnote[***]{$p < 0.001$}
}{                                                                        \FL
  & \multicolumn{2}{c}{Muži (N=371)} & \multicolumn{2}{c}{Ženy (N=352)} & \NN
  \multicolumn{1}{c}{(N=723)} %
                 &   N &     \% &   N &     \% & Pearsonův $\chi^2$       \ML
  \emph{Status/Majetek} \NN
  ~~Poptávaný    &  45 & 12,1\% &  77 & 21,9\% & 15,819\tmark[***] \NN
  ~~Nabízený     &  72 & 19,4\% &  57 & 16,2\% &  4,986~~~ \LL
}

V~tabulce~\ref{tab_status} je patrné, že ženy poptávají zdroje ve
formě statusu nebo majetku častěji než muži ($\chi^2$~=~16,466, df~=~2,
p~{\textless}~0,001), což potvrzuje hypotézu H4a.  Na rozdíl od
poptávky v~četnosti inzerátů nabízejících status nebo majetek není
mezi muži a~ženami statisticky signifikantní rozdíl. Hypotéza H4b se
tedy nepotvrdila.

\ctable[
  cap     = Poptávka a~nabídka závazku,
  caption = Procentuální rozložení inzerátů nabízejících nebo poptávajících parametry indikující závazek,
  label   = tab_commit,
  center,
  pos = htbp,
  botcap,
  notespar
]{lrrrrr}{
  \tnote[*]{$p < 0.05$}
  \tnote[**]{$p < 0.01$}
  \tnote[***]{$p < 0.001$}
}{                                                                        \FL
  & \multicolumn{2}{c}{Muži (N=371)} & \multicolumn{2}{c}{Ženy (N=352)} & \NN
  \multicolumn{1}{c}{(N=723)} %
                 &   N &     \% &   N &     \% & Pearsonův $\chi^2$       \ML
  \emph{Závazek} \NN
  ~~Poptávaný    & 156 & 42,1\% & 159 & 45,2\% & 1,509~~ \NN
  ~~Nabízený     &  61 & 16,4\% &  31 &  8,8\% & 9,959\tmark[**] \LL
}

V~tabulce~\ref{tab_commit} vidíme, že poptávka závazků mezi ženami
a~muži se v~seznamovacích inzerátech neliší.  Hypotéza H5a se
nepotvrdila. Co se týče nabídky závazku, tak muži nabízejí závazek
častěji než ženy ($\chi^2 $~=~9,959, df~=~2, p {\textless}
0.01). Hypotéza H5b byla potvrzena.

\ctable[
  cap     = Poptávka a~nabídka sociálních schopností,
  caption = Procentuální rozložení inzerátů nabízejících nebo poptávajících sociální schopnosti včetně smyslu pro humor,
  label   = tab_soc,
  center,
  pos = htbp,
  botcap,
  notespar
]{lrrrrr}{
  \tnote[*]{$p < 0.05$}
  \tnote[**]{$p < 0.01$}
  \tnote[***]{$p < 0.001$}
}{                                                                        \FL
  & \multicolumn{2}{c}{Muži (N=371)} & \multicolumn{2}{c}{Ženy (N=352)} & \NN
  \multicolumn{1}{c}{(N=723)} %
                 &   N &     \% &   N &     \% & Pearsonův $\chi^2$       \ML
  \emph{Sociální schopnosti} \NN
  ~~Poptávané    & 120 & 32,4\% & 170 & 48,3\% & 26,347\tmark[***] \NN
  ~~Nabízené     &  64 & 17,2\% &  61 & 17,4\% & 1,644~~~ \LL
}

Z~tabulky~\ref{tab_soc} lze zjistit, že existuje signifikantní rozdíl
v~poptávce sociálních schopností mezi muži a~ženami.  Ženy častěji
poptávají sociální schopnosti.  Z~celkového počtu poptávek sociálních
schopností ženami (N=170) jich více než polovina (N=93) obsahuje poptávku po
,,smyslu pro humor``.

\ctable[
  cap     = Poptávka partnerů s~dítětem nebo bez dítěte,
  caption = Procentuální zastoupení poptávky partnera s~dítětem/dětmi a~bez dítěte,
  label   = tab_child,
  center,
  pos = htbp,
  botcap,
  notespar
]{lrrrrr}{
  \tnote[*]{$p < 0.05$}
  \tnote[**]{$p < 0.01$}
  \tnote[***]{$p < 0.001$}
}{                                                                        \FL
  & \multicolumn{2}{c}{Muži (N=371)} & \multicolumn{2}{c}{Ženy (N=352)} & \NN
  \multicolumn{1}{c}{(N=723)} %
                 &   N &     \% &   N &     \% & Pearsonův $\chi^2$       \ML
  \emph{Partner s~dítětem} \NN
  ~~Poptávaný    &  30 &  8,1\% &   3 &  0,9\% & 20,070\tmark[***] \ML
  \emph{Partner bez~dítěte} \NN
  ~~Poptávaný    &   6 &  1,6\% &   5 &  1,4\% &  0,000~~~\NN
}

V~tabulce~\ref{tab_child} lze vidět, že muži signifikantně častěji
poptávají partnerku ,,i~s~dítětem``.  Obě dvě varianty
(,,i~s~dítětem``, ,,bez dítěte``) pozitivně korelují ($\rho$~=~0,178,
$\rho$~=~0,150, p~{\textless}~0,001) s~věkem muže, což si lze
vykládat předpokladem, že muži ve věku 30 až 40 let předpokládají, že
potenciální partnerka (i~mladší) již děti může mít, a~proto
explicitně ve svých inzerátech tento faktor zmiňují.

\ctable[
  cap     = Poptávka a~nabídka ostatních kategorií,
  caption = Procentuální zastoupení všech ostatních proměnných s~nesignifikantními rozdíly mezi pohlavími,
  label   = tab_rest,
  center,
  pos = htbp,
  botcap,
  notespar
]{lrrrrr}{
  \tnote[*]{$p < 0.05$}
  \tnote[**]{$p < 0.01$}
  \tnote[***]{$p < 0.001$}
}{                                                                        \FL
  & \multicolumn{2}{c}{Muži (N=371)} & \multicolumn{2}{c}{Ženy (N=352)} & \NN
  \multicolumn{1}{c}{(N=723)} %
                 &   N &     \% &   N &     \% & Pearsonův $\chi^2$       \ML
  \emph{Sexuální věrnost} \NN
  ~~Poptávaná    &  48 & 12,9\% &  50 & 14,2\% & 0,151 \NN
  ~~Nabízená     &  18 &  4,9\% &   8 &  2,3\% & 2,761 \ML
  \emph{Ostatní povahové} \NN
  \emph{a~interpersonální vlastnosti} \NN
  ~~Poptávané    &  41 & 11,1\% &  35 &  9,9\% & 0,133 \NN
  ~~Nabízené     &   9 &  2,4\% &  11 &  3,1\% & 0,120 \ML
  \emph{Koníčky: nákladné} \NN
  ~~Poptávané    &  15 &  4,0\% &  18 &  5,1\% & 0,261 \NN
  ~~Nabízené     &  28 &  7,5\% &  34 &  9,8\% & 0,776 \ML
  \emph{Koníčky: statusové} \NN
  ~~Poptávané    &   9 &  2,4\% &  13 &  3,7\% & 0,601 \NN
  ~~Nabízené     &  14 &  3,8\% &  15 &  4,3\% & 0,021 \ML
  \emph{Koníčky: fyzické} \NN
  ~~Poptávané    &  59 & 15,9\% &  72 & 20,5\% & 2,541 \NN
  ~~Nabízené     &  93 & 25,1\% &  73 & 20,8\% & 2,326 \ML
  \emph{Koníčky: ostatní} \NN
  ~~Poptávané    &  31 &  8,4\% &  36 & 10,2\% & 0,546 \NN
  ~~Nabízené     &  40 & 10,8\% &  48 & 13,6\% & 1,123 \ML
}

Tabulka~\ref{tab_rest} obsahuje všechny ostatní sledované proměnné,
u~kterých nebyl zjištěn signifikantní rozdíl mezi pohlavími.

\ctable[
  cap = Pořadí poptávaných kategorií,
  caption = Seznam inzerenty poptávaných kategorií seřazených dle celkové četnosti,
  label   = tab_sought,
  center,
  pos = htbp,
  botcap,
  notespar
]{lrrr}{}{                                                                        \FL
          & \multicolumn{1}{c}{Pořadí muži} & \multicolumn{1}{c}{Pořadí ženy} & \multicolumn{1}{c}{Pořadí celkem} \ML
  \emph{Závazek} & 1 & 2 & 1 \NN
  \emph{Sociální a~interpers. schopnosti} & 2 & 1 & 2 \NN
  \emph{Věkový rozdíl} & 4 & 3 & 3 \NN
  \emph{Fyzická atraktivita} & 3 & 8 & 4 \NN
  \emph{Koníčky: fyzické} & 5 & 5 & 5 \NN
  \emph{Status/Majetek} & 7 & 4 & 6 \NN
  \emph{Sexuální věrnost} & 6 & 6 & 7 \NN
  \emph{Ostatní povahové vlastnosti}  & 8 & 10 & 8 \NN
  \emph{Koníčky: ostatní} & 9 & 9 & 9 \NN
  \emph{Fotografie} & 11 & 7 & 10 \NN
  \emph{Koníčky: nákladné} & 12 & 12 & 11--12 \NN
  \emph{Partner s~dítětem} & 10 & 15 & 11--12 \NN
  \emph{Výška} & 15 & 11 & 13 \NN
  \emph{Koníčky: statusové} & 13 & 13 & 14 \NN
  \emph{Partner bez~dítěte} & 14 & 14 & 15 \NN
}

Z~tabulky \ref{tab_sought} lze zjistit, že nejčastěji vyhledávanou
vlastností, o~kterých se v~inzerátech zmiňují muži, je budoucí závazek
ženy.  Na druhém místě se umístily sociální schopnosti.  Na třetím
místě se umístil požadavek na fyzickou atraktivitu.  Ženami jsou
nejčastěji vyhledávané sociální a~interpersonální schopnosti,
následované poptávkou závazku.  Na třetím místě se umístila kategorie
vyššího věku muže.  Celkově je pořadí poptávaných ka\-te\-gorií
následující: 1. \emph{Závazek}, 2. \emph{Sociální a~interpersonální
  schopnosti}, 3. \emph{Věkový rozdíl partnera}.

\ctable[
  cap = Pořadí nabízených kategorií,
  caption = Seznam inzerenty nabízených kategorií seřazených dle celkové četnosti,
  label   = tab_offer,
  center,
  pos = htbp,
  botcap,
  notespar
]{lrrr}{}{                                                                        \FL
  & \multicolumn{1}{c}{Pořadí muži} & \multicolumn{1}{c}{Pořadí ženy} & \multicolumn{1}{c}{Pořadí celkem} \ML
  \emph{Koníčky: fyzické} & 1 & 2 & 1 \NN
  \emph{Fyzická atraktivita} & 2--3 & 1 & 2 \NN
  \emph{Status/Majetek} & 2--3 & 4 & 3 \NN
  \emph{Sociální schopnosti} & 4 & 3 & 4 \NN
  \emph{Závazek} & 5 & 7 & 5 \NN
  \emph{Koníčky: ostatní} & 6 & 5 & 6 \NN
  \emph{Koníčky: nákladné} & 7 & 6 & 7 \NN
  \emph{Koníčky: statusové} & 10 & 8 & 8 \NN
  \emph{Výška} & 8 & 11 & 9 \NN
  \emph{Sexuální věrnost} & 9 & 10 & 10 \NN
  \emph{Ostatní povahové vlastnosti} & 11 & 9 & 11 \LL
}

Tabulka~\ref{tab_offer} obsahuje seznam nejčastěji nabízených
kategorií.  Muži nejčastěji nabízí fyzicky zaměřené koníčky,
majetek/status a~závazek.  Ženy se snaží muže zaujmout fy\-zickou
atraktivitou, fyzicky zaměřenými koníčky a~nabídkou sociálních
schopností.  Celkově je pořadí nabízených kategorií následující:
1.~\emph{Koníčky: fyzické}, 2.~\emph{Fyzická atraktivita},
3.~\emph{Status/majetek}.

Vzhledem k~umístění sociálních schopností, které jsou vyhledávány
u~mužů na druhém místě, u~žen dokonce na prvním místě a~celkově na
druhém místě, byla potvrzena hypotéza H6a.  Závazek byl vyhledáván muži
na prvním místě, ženami na druhém místě a~celkově se tato kategorie
umístila na první místě.  Hypotéza H6b byla také potvrzena.

\section[Diskuze]{Diskuze}

Cílem této práce bylo zodpovědět otázku, zda-li se
evolučně-psychologické principy výběru partnera promítají do online
seznamovacích inzerátů.  Na základě zkoumaných hypotéz a~na základě
zkoumaného vzorku je možné říci, že tyto principy se uplatňují i~při
online seznamování.

Zjištěná důležitost fyzické atraktivity (v~celkové poptávce na čtvrtém
místě, v~celkové nabídce na místě druhém) pro prvotní fázi vztahu
potvrzuje i~předpoklady Stimulus fáze Mursteinovy
\citeyearpar{Murstein1970} teorie S--V--R.

Výsledky výzkumu potvrdily všechny předložené evolučně-psychologické
hypotézy, kromě hypotézy H4b.  Ta zkoumala jestli muži více než ženy
v~seznamovacích inzerátech nabízejí charakteristiky dokládající jejich
možnosti poskytnout budoucímu partnerovi zdroje ve formě majetku nebo
statusu.

Možná příčina nepotvrzení této hypotézy může ležet v~proměně západní
společnosti oproti stavu předchozího zkoumání seznamovacích inzerátů,
jak je prezentují \citet{BarrettDunbarLycett2007}.  Do proměnné
status/majetek bylo dle jejich vzoru kódováno vysokoškolské vzdělaní
(i~studium).  V~současné společnosti se zvyšuje počet vysokoškolských
studentů a~absolventů.  Zvyšuje se i poměrné zastoupení žen mezi
studenty a~ještě výrazněji i~mezi absolventy \citep{CSU2010}.  Celkové
zastoupení žen mezi studenty bylo v~roce 2008 51\% a~mezi absolventy
dokonce 57\% \citep{CSU2010}.  Používání internetu také pozitivně
koreluje s~dosaženým vzděláním, nejvíce je internet využíván
vysokoškoláky a~vysokoškolskými studenty \citep{GalaczSmahel2007}.
Data získaná v~tomto výzkumu neobsahují data o~nejvyšším dosaženém
vzdělání, je možné předpokládat, že i~internetové seznamky použité
v~tomto výzkumu budou více využívány lidmi s~vyšším vzděláním, což
mohlo ovlivnit výsledky výzkumu.  Tuto teorii podporuje i~fakt, že
z~57 inzerátů podaných ženami, které obsahovaly indikátor
status/majetek, jich více než polovina (30) byla založena pouze na
vysokoškolském vzdělání (studentky nebo absolventky).  U~mužů poměr
inzerátů, které jsou založeny pouze na vysokoškolském vzdělání, není
tak velký.  Jen 13 inzerátů ze 72 bylo kódováno v~kategorii
status/majetek na základě vysokoškolského vzdělání.  Po vyřazení
klíčových slov týkajících se vysokoškolského vzdělání z~kategorie
status/majetek by muži nabízeli častěji než ženy klíčová slova
v~kategorie status/majetek.

Další zajímavý rozpor byl zjištěn mezi rozdíly mužů a~žen při nabídce
a~poptávce fyzické atraktivity a~nabídce a~poptávce fotografií druhé
osoby.  Možné vysvětlení tohoto rozporu může ležet ve větší
zranitelnosti žen při seznamování ,,na slepo``. Mužská sexuální agrese
je mnohem častější a~je možné, že se ženy snaží chránit své soukromí
do doby, než ohodnotí zájemce o~seznámení na základě odpovědi na
inzerát a~případně i~jeho přiložené fotografie.

Zjištěný rozdíl (v~hypotézách nezkoumaný) v~poptávce partnerů
i~s~dítětem mezi muži a~ženami nemá oporu v~teorii je pravděpodobně
způsoben vyřazením inzerátů, kde inzerent deklaroval, že již děti má.
Z~tohoto rozdílu není možno vyvozovat žádné závěry.  Bylo by možné
provést rozšířený průzkum, ve kterém by byl detailněji zjišťován počet
dětí inzerenta a~jeho požadavky na děti potenciální partnerky.

Další možná cesta zkoumání seznamování na internetu kvantitativní
metodou může být v~rozdílném přístupu k~získaným datům.  Inzeráty by
bylo možné analyzovat také rozdělené do věkových kategorií.  V~takovém
případě by bylo možné do studie zahrnout i~inzerenty mladší 20 let
a~starší 40 let.

V~následující části jsou uvedeny poznatky, které bylo možno zjistit
jen při ručním kódování a~po překódování do kategorií pro
kvantitativní analýzu byly ztraceny.

Pokud muži uvedou v~inzerátu barvu vlasů nebo barvu očí, bývá to ve
většině případů spojeno s~tím, že také uvádějí další indikátory
fyzické atraktivity, jako: ,,štíhlý`` nebo ,,pohledný``. U~žen takto
jasné rozlišení není a~i~ženy, které uvedou, že jsou ,,plnoštíhlé``
či ,,baculky``, připojují k~popisu barvu vlasů, barvu očí nebo
obojí. Nabízí se myšlenka, že ženy nabídkou barvy vlasů nebo očí
kompenzují vyšší váhu, která dle evoluční psychologie nebývá
preferována.  Zajímavé také je, že tato nabídka ovšem není vůbec
reflektována na straně poptávky.  Ve vybraném vzorku inzerátů podaných
muži ani ženami nebyl ani jediný, který by poptával žádanou barvu
vlasů nebo barvu očí.

Ženy v~inzerátech používaly mnohem méně často slovní spojení ,,vážný
vztah``.  Důvodů pro toto může být více.  Inzeráty, které byly použity
pro tento výzkum, již byly umístěny v~kategorii vážné seznámení.  Je
možné, že ženy nepociťují takovou potřebu znovu tento fakt
zdůrazňovat.  Muži na druhou stranu mnohdy zařazení inzerátů
,,ignorují`` a~i~v~ka\-te\-gorii vážné seznámení hledají pouze sexuální
partnerku.  Další důvod může ležet již v~samotném faktu, že ženy by
měly více vyhledávat závazek, protože u~nich samotných se vzhledem
k~velké investici na zplození a~výchovu potomka vlastní závazek
implicitně předpokládá \citep{Trivers1972}.  Další důvod neuvádění
požadavku na ,,vážný vztah`` může ležet v~oblasti sociální
žádoucnosti, resp. ne-žádoucnosti, kdy samotná poptávka závazku může
mít účinek ve snížení počtu odpovědí na inzerát.

Na závěr diskuze bych rád uvedl zajímavý inzerát, který by mohl být
inspirací pro další výzkum:

,,Ahojky, hledám někoho, komu nebude vadit, že chci založit rodinu. I
kdyby to měl být třeba jen přátelský vztah, hledám někoho, komu nevadí
můj věk, chci miminko a sama to zkrátka nejde :-) Zaručuji
stoprocentní zdraví, můžu ti i donést výsledky krevních testů....a
totéž také hledám :-)``

Žena či dívka v~tomto inzerátu nabízí seznámení pouze za účelem početí
dítěte.  V~tomto případě je tento neobvyklý požadavek ještě eskalován
tím, že inzerentka ve strukturované části inzerátu uvádí, že je ve
věku 21 let. Bylo by jistě zajímavé zjistit individuální motivy
k~takovému jednání i~s~ohledem na to, že se podobné případy objevují
i~v~populární kultuře (např. film Big Lebowski a~další).

\chapter{Závěr}

Tato práce se zabývala kvantitativním ověřováním hypotéz, které
vyplývají z~evoluční psychologie lidské volby partnera.  Cílem bylo
zjistit, zda-li se předpoklady evoluční psychologie potvrdí
i~v~prostředí internetu.

Ze zkoumaných proměnných byly zjištěny rozdíly v~poptávce fyzické
atraktivity muži a~v~nabídce fyzické atraktivity ženami.  Dále byl
zjištěn rozdíl v~vyšší poptávce mužské výšky ženami a~vyšší nabídce
výšky muži.  Z biologického hlediska byl také ověřen před\-po\-klad, že
muži budou vyhledávat mladší partnerky a~ženy starší partnery.

Další signifikantní rozdíly byly zjištěny mezi poptávkou a~nabídkou
proměnné status/majetek.  Ženy častěji než muži poptávají klíčová
slova indikující majetek nebo status, ale nebyl zjištěn statisticky
signifikantní rozdíl mezi muži a~ženami v~nabídce majetku nebo statusu.

Předpoklad, že ženy budou více než muži poptávat závazek, také nebyl
potvrzen.  V~poptávce závazku mezi pohlavími neexistuje statisticky
významný rozdíl.  Doplňkově muži nabízejí závazek častěji než ženy.

Ve výsledcích byla potvrzena důležitost sociálních schopností a~smyslu
pro humor pro seznamování.  Klíčová slova týkající se této kategorie
se v~inzerátech obou pohlaví objevovala jako jedny z~nejčastějších.

Pokud by bylo možné získat po domluvě s~provozovatelem online seznamky
data týkající se počtu odpovědí na inzerát, tak by bylo možné výsledky
tohoto výzkumu propojit s~analýzou reálné situace.  Tedy zda-li se
nabízené charakteristiky skutečně promítají do výsledného počtu
odpovědí na inzerát, a~jaké další faktory tento počet ovlivňují.

Výsledky tohoto výzkumu je také možné využít v~poradenské praxi.
Klientovi, ne\-ús\-pě\-š\-ném v~seznamovacím procesu na internetu by bylo
možné poradit úspěšnější strategii psaní seznamovacího inzerátu a~tím
předejít dalším frustracím.  Prozkoumání možností, jak zlepšit
úspěšnost seznamovacího inzerátu, by mohlo být zajímavým tématem pro
kvalitativní studii.

\clearpage
\singlespacing
\bibliographystyle{apa-good}
%\bibliographystyle{abbrvnat}
\bibliography{Literatura}

\end{document}

% LocalWords:  Feelovi EllisonEtAl Slamenik Mursteinovu Bowlby GuerreroAndersen
% LocalWords:  Kelley personal relationships Rican LangmeierKrejcirova Vyrost
% LocalWords:  BurmanPelaezSanchez Murstein Value rozlady Bateman Drosophila vs
% LocalWords:  melanogaster Ridley Batemanem Batemanův protistrategii Dawkins
% LocalWords:  BussSchmitt Batemanova Barber runaway GangestadBuss Evolutionary
% LocalWords:  DunbarBarrett Handbook culture nature tabula Mameli Triversovy
% LocalWords:  polyandrynní BussEtAl Hledam
